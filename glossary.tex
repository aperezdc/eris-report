% vim: ft=tex spell spelllang=en
%
% Glossary/Acronyms
%

\newglossaryentry{transpiler}{
	name=transpiler,
	description={Type of compiler that takes source code of a programming
		language as its input, and produces a different source code as
		output, usually in a different programming language.  Also known
		as \emph{source-to-source compiler}, or \emph{transcompiler}},
}

\newglossaryentry{lua-users-wiki}{
	name={Lua-Users Wiki},
	description={Community operated \emph{wiki} site which contains resources
		for Lua development, written by users of the programming language
		themselves. URL address: \url{http://lua-users.org/wiki}},
}

\newglossaryentry{data-deduplication}{
	name={data deduplication},
	description={any technique which eliminates duplicate copies of repeating
		data in order to improve storage utilization},
}

\newglossaryentry{gls-TUE}{
	name={Type Unit Entry},
	description={A particular kind of DWARF DIE that contains information
		about a data type},
}
\newacronym[see={[Glossary:]{gls-TUE}}]{TUE}{TUE}
	{Type Unit Entry\glsadd{gls-TUE}}


\newacronym{ELF}{ELF}{Executable and Linkable Format}
\newacronym{DWARF}{DWARF}{Debugging With Attributed Record Formats}
\newacronym{JIT}{JIT}{Just-In-Time}
\newacronym{FFI}{FFI}{Foreign Function Interface}
\newacronym{DIE}{DIE}{Debugging Information Entry}
\newacronym{CU}{CU}{Compilation Unit}
\newacronym{FDL}{FDL}{Free Documentation License}
\newacronym{API}{API}{Application Programmin Interface}
\newacronym{PIC}{PIC}{Position-Independent Code}

%
% Those are just simple command-abbreviations to format pieces of text which
% should be always displayed with the same formatting. Using a macro ensures
% that, and makes it easier to come back here and change the formatting for
% all occurrences, if needed.
%
\def\Eris*{\textsc{\sffamily Eris}}

