% vim: ft=tex spell spelllang=en
%
% Glossary/Acronyms
%

\newglossaryentry{transpiler}{
	name=transpiler,
	description={Type of compiler that takes source code of a programming
		language as its input, and produces a different source code as
		output, usually in a different programming language.  Also known
		as \emph{source-to-source compiler}, or \emph{transcompiler}},
}

\newglossaryentry{lua-users-wiki}{
	name={Lua-Users Wiki},
	description={Community operated \emph{wiki} site which contains resources
		for Lua development, written by users of the programming language
		themselves. URL address: \url{http://lua-users.org/wiki}},
}

\newglossaryentry{pascal}{
	name={Pascal},
	description={
		Procedural programming language designed in the late 60s by Niklaus Wirth
		to encourage good programming practices
	},
}

\newglossaryentry{name-mangling}{
	name={name mangling},
	description={
		Technique used generate unique names for programming entities, usually by
		encoding additional information about the entity in its name
	},
}

\newglossaryentry{object-oriented}{
	name={object oriented},
	description={
		Programming paradigm based on the concept of \emph{objects}, which are
		data structures that encapsulate both data, and its behavior
	},
}

\newglossaryentry{data-deduplication}{
	name={data deduplication},
	description={any technique which eliminates duplicate copies of repeating
		data in order to improve storage utilization},
}

\newglossaryentry{flexible-array-member}{
	name={flexible array member},
	description={
		Feature introduced in the C99 standard of the C programming language
		which allows the last member of a \Mc:struct: to be an array of an
		unspecified dimension. Space needed by the array does not contribute
		to the size of the \Mc:struct: type, and must be manually accounted
		for when allocating the \Mc:struct: from the heap
	},
}

\newglossaryentry{emulation}{
	name={emulation},
	description={
		Piece of hardware or software that enables one computer system (called
		the \emph{host}) to behave like another computer system (called the
		\emph{guest}) which enables the host system to run software or use
		peripheral devices designed for the guest system
	},
}

\newglossaryentry{Toyota}{
	name={Toyota},
	description={Japanese car manufacturer},
}

\newglossaryentry{gls-TUE}{
	name={Type Unit Entry},
	description={A particular kind of DWARF DIE that contains information
		about a data type},
}
\newacronym[see={[Glossary:]{gls-TUE}}]{TUE}{TUE}
	{Type Unit Entry\glsadd{gls-TUE}}

\newglossaryentry{gls-ISA}{
	name={Instruction Set Architecture},
	description={
		Part of the computer architecture related to programming, including
		the native data types, instructions, registers, addressing modes,
		memory architecture, interrupt and exception handling, and external
		I/O. An ISA includes a specification of the machine language, and
		the native commands implemented by a particular processor.
	},
}
\newacronym[see={[Glossary:]{gls-ISA}}]{ISA}{ISA}
	{Instruction Set Architecture\glsadd{gls-ISA}}


\newacronym{ELF}{ELF}{Executable and Linkable Format}
\newacronym{DWARF}{DWARF}{Debugging With Attributed Record Formats}
\newacronym{JIT}{JIT}{Just-In-Time}
\newacronym{FFI}{FFI}{Foreign Function Interface}
\newacronym{DIE}{DIE}{Debugging Information Entry}
\newacronym{CU}{CU}{Compilation Unit}
\newacronym{FDL}{FDL}{Free Documentation License}
\newacronym{API}{API}{Application Programming Interface}
\newacronym{PIC}{PIC}{Position-Independent Code}
\newacronym{VLA}{VLA}{Variable-Length Array}
\newacronym{PUC-Rio}{PUC-Rio}{Pontifícia Universidade Católica do Rio de Janeiro}
\newacronym{VM}{VM}{Virtual Machine}

%
% Those are just simple command-abbreviations to format pieces of text which
% should be always displayed with the same formatting. Using a macro ensures
% that, and makes it easier to come back here and change the formatting for
% all occurrences, if needed.
%
\def\Eris*{\textsc{\sffamily Eris}}
