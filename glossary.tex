% vim: ft=tex spell spelllang=en
%
% Glossary/Acronyms
%

\newglossaryentry{LuaJIT}{
	name=LuaJIT,
	description={
		Just-In-Time compiler (\gls{JIT}) for the Lua programming language.
		It is a third-party, independent implementation of the Lua VM created
		and maintained by Mike Pall, who allegedly was born in planet Krypton.
		Available at \url{http://luajit.org}
	},
}

\newglossaryentry{transpiler}{
	name=transpiler,
	description={Type of compiler that takes source code of a programming
		language as its input, and produces a different source code as
		output, usually in a different programming language.  Also known
		as \emph{source-to-source compiler}, or \emph{transcompiler}
	},
}

\newglossaryentry{lua-users-wiki}{
	name={Lua-Users Wiki},
	description={Community operated \emph{wiki} site which contains resources
		for Lua development, written by users of the programming language
		themselves. URL address: \url{http://lua-users.org/wiki}
	},
}

\newglossaryentry{pascal}{
	name={Pascal},
	description={
		Procedural programming language designed in the late 60s by Niklaus Wirth
		to encourage good programming practices
	},
}

\newglossaryentry{name-mangling}{
	name={name mangling},
	description={
		Technique used generate unique names for programming entities, usually by
		encoding additional information about the entity in its name
	},
}

\newglossaryentry{object-oriented}{
	name={object oriented},
	description={
		Programming paradigm based on the concept of \emph{objects}, which are
		data structures that encapsulate both data, and its behavior
	},
}

\newglossaryentry{data-deduplication}{
	name={data deduplication},
	description={Any technique which eliminates duplicate copies of repeating
		data in order to improve storage utilization
	},
}

\newglossaryentry{flexible-array-member}{
	name={flexible array member},
	description={
		Feature introduced in the C99 standard of the C programming language
		which allows the last member of a \Mc:struct: to be an array of an
		unspecified dimension. Space needed by the array does not contribute
		to the size of the \Mc:struct: type, and must be manually accounted
		for when allocating the \Mc:struct: from the heap
	},
}

\newglossaryentry{emulation}{
	name={emulation},
	description={
		Piece of hardware or software that enables one computer system (called
		the \emph{host}) to behave like another computer system (called the
		\emph{guest}) which enables the host system to run software or use
		peripheral devices designed for the guest system
	},
}

\newglossaryentry{metaprogramming}{
	name={metaprogramming},
	description={
		Writing of computer programs which are able to read, generate, analyze
		or transform other programs, or even modify themselves while running
	},
}

\newglossaryentry{memoization}{
	name={memoization},
	description={
		Optimization technique which stores the results of expensive function
		calls, and returns the previously calculated value when the same inputs
		occur again
	},
}

\newglossaryentry{dynamic-programming}{
	name={dynamic programming},
	description={
		Problem solving method —and programming technique— which solves a
		complicated problem by breaking it up in smaller problems in a
		recursive manner
	}
}

\newglossaryentry{dynamic-dispatch}{
	name={dynamic dispatch},
	description={
		Process of selecting a concrete implementation of a polymorphic
		method (or function) at runtime. It is typically used in object
		oriented languages when different classes contain different
		implementations of the same method due to inheritance
	}
}

\newglossaryentry{fibonacci-number}{
	name={Fibonacci number},
	description={
		Number from the sequence $1, 1, 2, 3, 5, 8, 13, ...$, given by the
		recurrence relation $F_n = F_{n-1} + F_{n-2}$, with $F_1 = 1$, and
		$F_2 = 1$, as defined by the Italian mathematician Leonardo Fibonacci
	},
}

\newglossaryentry{first-class-value}{
	name={first--class value},
	description={
		In programming language design, an entity which supports all the
		operations generally available to other entities of a laguage,
		typically: being passed as a parameter, returned from a function,
		and assigned to a variable
	},
}

\newglossaryentry{closure}{
	name={closure},
	description={
		Technique for implementing lexically scoped name binding in languages
		with first--class functions
	},
}

\newglossaryentry{constructor}{
	name={constructor},
	description={
		Special type of subroutine in a program which is called to create an
		object, and performing any initialization needed before the object can
		be used
	},
}

\newglossaryentry{refcounting}{
	name={reference counting},
	description={
		Technique of storing the number of references to an object, block of
		memory, disk space, or any other resource, which allows tracking
		whether the resource is in use by others
	},
}

\newglossaryentry{backronym}{
	name={backronym},
	description={
		A \emph{backward acronym} is an acronym constructed in reverse, by
		creating a new phrase to fit an existing word, name, or acronym
	},
}

\newglossaryentry{Toyota}{
	name={Toyota},
	description={Japanese car manufacturer},
}

\newglossaryentry{gls-ABI}{
	name={Application Binary Interface},
	description={
		Interface between two program modules, one of which is
		usually a library or the operating system, at the level of machine
		code. An ABI determines such details as how functions are called,
		and how parameters are passed to them
	},
}
\newacronym[see={[Glossary:]{gls-ABI}}]{ABI}{ABI}
	{Application Binary Interface\glsadd{gls-ABI}}

\newglossaryentry{gls-TUE}{
	name={Type Unit Entry},
	description={A particular kind of DWARF DIE that contains information
		about a data type},
}
\newacronym[see={[Glossary:]{gls-TUE}}]{TUE}{TUE}
	{Type Unit Entry\glsadd{gls-TUE}}

\newglossaryentry{gls-ISA}{
	name={Instruction Set Architecture},
	description={
		Part of the computer architecture related to programming, including
		the native data types, instructions, registers, addressing modes,
		memory architecture, interrupt and exception handling, and external
		I/O. An ISA includes a specification of the machine language, and
		the native commands implemented by a particular processor.
	},
}
\newacronym[see={[Glossary:]{gls-ISA}}]{ISA}{ISA}
	{Instruction Set Architecture\glsadd{gls-ISA}}

\newglossaryentry{gls-GC}{
	name={Garbage Collection},
	description={
		Method of automatic memory management, in which a \emph{garbage
		collector} tries to reclaim “garbage” (memory occupied by data no
		longer in use by the program) with a certain periodicity
	},
}
\newacronym[see={[Glossary:]{gls-GC}}]{GC}{GC}
	{Garbage Collection\glsadd{gls-GC}}

\newglossaryentry{gls-TAP}{
	name={Test Anything Protocol},
	description={
	},
}
\newacronym[see={[Glossary:]{gls-TAP}}]{TAP}{TAP}
	{Test Anything Protocol\glsadd{gls-TAP}}

\newacronym{ELF}{ELF}{Executable and Linkable Format}
\newacronym{DWARF}{DWARF}{Debugging With Attributed Record Formats}
\newacronym{JIT}{JIT}{Just-In-Time}
\newacronym{FFI}{FFI}{Foreign Function Interface}
\newacronym{DIE}{DIE}{Debugging Information Entry}
\newacronym{CU}{CU}{Compilation Unit}
\newacronym{FDL}{FDL}{Free Documentation License}
\newacronym{API}{API}{Application Programming Interface}
\newacronym{PIC}{PIC}{Position-Independent Code}
\newacronym{VLA}{VLA}{Variable-Length Array}
\newacronym{PUC-Rio}{PUC-Rio}{Pontifícia Universidade Católica do Rio de Janeiro}
\newacronym{VM}{VM}{Virtual Machine}
\newacronym{IRC}{IRC}{Internet Relay Chat}
\newacronym{OSI}{OSI}{Open Source Initiative}

%
% Those are just simple command-abbreviations to format pieces of text which
% should be always displayed with the same formatting. Using a macro ensures
% that, and makes it easier to come back here and change the formatting for
% all occurrences, if needed.
%
\def\Eol*{\textsc{\sffamily Eöl}}
