% vim: ft=tex ts=2 sw=2 spell spelllang=en
\chapter*{Summary}

The objective of this project is to implement an automated mechanism that,
using the DWARF debugging information from ELF shared objects, allows the Lua
virtual machine to call native functions from shared objects implemented in
the C programming language. The process is automatic, in the sense that the
user does not need to write code to convert values passed between Lua and the
invoked C functions, and the C functions will behave essentially like Lua from
the user point of view. The ultimate goal is to allow transparent usage of
existing C libraries from Lua.

Lua has been chosen because it provides a clean C interface to its \gls{VM},
which has been designed from the ground up to be embedded in larger projects.
The implementation is also compact (under 16.000 lines of code), which makes
it feasible to gain in-depth knowledge of its innerworkings in a relatively
short time. Lua has also grown in popularity in the last years as its adoption
has skyrocketed in the game industry.

The reason to focus on the combination of debugging information in DWARF
format contained in ELF shared objects is that they are a widespread, standard
configuration used by the majority of contemporary Unix-like operating
systems. The target system during development has been a GNU/Linux system
running on the Intel x86\_64 architecture, which also uses the aforementioned
configuration, though provisions are to be included in the design to ease
future porting efforts for other platforms.

In order to validate the correctness of the implementation, an automated test
suite was also developed. Unit tests were used also as regression tests, to
ensure that modifications to the system did not introduce programming errors
in the implementation.
