% vim: ft=tex spell spelllang=en ts=2 sw=2

\cleardoublepage
\setchaptertoc
\chapter{Analysis \& Design}

This chapter is a tour through the architecture of the developed software
solution, analyzing relevant decisions taken that gave it its final shape.
\afterintro

\section{Naming}

In the Lua community there is a certain tradition of naming projects after
celestial bodies, or terms related to them —after all, Lua means \emph{moon}
in Portuguese—, but unfortunately the name initially chosen for the project
was Eris —a dwarf planet, neither a planet nor a moon— was already being used
by another Lua-related project\footnote{The Eris persistence system,
\url{https://github.com/fnuecke/eris}}. A closer inspection showed that other
dwarf planet names were already in use for software projects, so in the end
it was needed to draw inspiration from a different area.

Eöl, also known as “The Dark Elf”, is a fictional character in
J. R. R. Tolkien's Middle-earth legendarium, who is said to be the elf with
closest relationships with dwarves, and one of the first able to speak their
language.  \Eol* can also be an \gls{backronym} for “ELF Object Loader”,
which describes well the purpose of the developed solution.


\section{Overview}
	\label{sec:design-overview}

The main components of \Eol* are shown in \autoref{fig:eol-architecture}.

The design of the system revolves around \textsf{Type Information}: it
describes native types in detail, and it is used by all the other components
in different ways to provide their functionality. Its importance should not be
surprising, because the ultimate goal of \Eol* is to allow seamless invocation
of native functions which, being close to the bare metal, always conform to
strict \gls{ABI} specifications. While starting execution of a native function
is as simple as generating a jump machine instruction to its start address in
memory, the function will only behave as expected if the data it uses —its
parameters, space for return values, etc.— is laid out in memory exactly in
the way its machine code expects it to be. In turn, this layout depends on the
types of the values passed to and from the function.

\tikzstyle{component} = [
  draw=black,
  thick,
  fill=green!10,
  rectangle,
  text centered,
  minimum height=2em,
  text width=6em,
  rounded corners,
  drop shadow,
]
\tikzstyle{uses} = [
  draw,
  very thick,
  >=triangle 45,
  ->,
  dashed,
]
\tikzstyle{contains} = [
  draw,
  thick,
  >=triangle 45,
  -*,
]

\begin{figure}
  \centering
  \begin{tikzpicture}[node distance=2cm]

    \node[component] (library) {Library};
    \node[component] (typecache) [above of=library]   {Type Cache};
    \node[component] (ctype)     [right=1cm of library]   {CType};
    \node[component] (function)  [right=1cm of ctype]     {Function};
    \node[component] (variable)  [right=1cm of function]  {Variable};
    \node[component] (typeinfo)  [above of=function]  {Type Information};
    \node[datain]    (dwarf)     [above=1cm of typecache]
                                  {DWARF debugging information};

    \node (luadata) [below right of=ctype] {Visible in Lua as userdata};
    \node (elf)     [above=0cm of dwarf]   {ELF shared object};

    \path[uses] (function) -- (typeinfo);
    \path[uses] (variable) -- (typeinfo);
    \path[uses] (ctype) -- (typeinfo);
    \path[uses] (typecache) -- (dwarf);
    \path[contains] (library) -- (typecache);
    \path[contains] (typecache) -- (typeinfo);

    \begin{pgfonlayer}{background}
      \node[datablob] (elfbox) [fit=(dwarf) (elf), drop shadow] {};
      \node[fill=yellow!20, rectangle, rounded corners] (wrappers)
      [fit=(library) (variable) (function) (luadata)] { };
    \end{pgfonlayer}
  \end{tikzpicture}
  \caption{Architecture of \Eol*.}
  \label{fig:eol-architecture}
\end{figure}

There are four components which form part of the interface to Lua (as
specified in \autoref{sec:design-lua-api}):

\begin{itemize}

	\item \textsf{Library} (\autoref{sec:eol-api-library-t}) represents a loaded
	ELF library. It is responsible of accessing the DWARF debugging information,
	and looking up values of the other types.

	\item \textsf{CType} (\autoref{sec:eol-api-ctype-t}) represents a native
	data type. It is responsible for providing information about the represented
	native type, and for creating native values of the represented native type
	from Lua. The types available to C programs are supported, hence the name.

	\item \textsf{Function} (\autoref{sec:eol-api-function-t}) represents a
		fragment of native code contained in a library, which can be invoked as
		a function. It is responsible for performing calls into native code
		from Lua.

	\item \textsf{Variable} (\autoref{sec:eol-api-variable-t}) represents a
		variable from a library. It is responsible for allowing reading and
		writing its value from Lua, performing conversions as needed.

\end{itemize}

Looking up type information involves reading the DWARF debugging information
from disk and decoding it appropriately. In order to avoid repeatedly reading
the debugging information from disk to construct new \textsf{Type Information}
values, each \textsf{Library} makes use of a \textsf{Type Cache} which keeps
the information in memory. An additional benefit of the cache is that it
allows reusing the \textsf{Type Information}: many DWARF \gls{DIE}s contain
references to others\todo{Got time? Add a diagram with an example}, and the
cache can be queried to determine whether a referenced DIE has been already
turned into \textsf{Type Information}, and use the data from the cache
instead.


\section{Interaction With the Lua GC}
	\label{sec:design-gc-interaction}

Userdata values are subject to Lua's \gls{GC} (c.f.
\autoref{sec:userdata-lua-custom-allocator}), which poses a problem for the
\textsf{Library} userdata: if the Lua VM does not keep an active reference to
a \textsf{Library} value, the GC will consider it to be garbage, and will
deallocate it while it may be still referenced by other resources. In
particular, a \textsf{Library} cannot be unloaded while there is any
\textsf{CType}, \textsf{Function}, or \textsf{Variable} userdata which belong
to the library being used from Lua. This kind of situation can be triggered by
the following simple sequence of events, illustrated by
\autoref{lst:library-gc-issue}:

\begin{listing}[ht]
	\begin{luacode}
		function loadfunction(libname, funcname)
		  local eol = require("eol")
		  local lib = eol.load(libname)
		  return lib[funcname]
		end
		-- Obtain an userdata for the add() function from libtest.so
		add = loadfunction("libtest", "add")
		-- Crashes if the GC has already collected the library.
		print(add(6, 5))
	\end{luacode}
	\caption{Lua example which makes a \textsf{Library} subject to GC}
	\label{lst:library-gc-issue}
\end{listing}

\begin{enumerate}

	\item A library is loaded and returned to Lua as a \textsf{Library}
		userdata. The userdata is assigned to a temporary variable (e.g.
		a \Mlua|local| variable inside a \Mlua|function|) which eventually
		will go out of scope.
	\item A \textsf{Function} userdata is obtained for a native function
		contained by the library.
	\item The GC determines that the \textsf{Library} userdata is garbage,
		and frees the resources used by it. This unloads the library.
	\item At this point, invoking the function crashes the program because
		its machine code, contained in the library, is no longer loaded in
		memory.

\end{enumerate}

The solution for this problem is to use \gls{refcounting}, to ensure that the
libraries are kept loaded while needed: each active userdata value of type
\textsf{Function}, \textsf{CType}, or \textsf{Variable} contributes to the
reference count. This way, a library is unloaded only when its reference count
reaches zero.


\section{Module API}
	\label{sec:design-lua-api}

The \gls{API} exposed by the \Eol* module to the Lua world is loosely modelled
after the one provided by the LuaJIT FFI module~\cite{lj-ffi-api} —also
implemented by the standalone \verb|luaffi| module~\cite{luaffi}—, with some
functions even having the same names and semantics, and others differing where
appropriate. For example, \Eol* does not need to provide a function to parse
C-like declarations because the type information is obtained from the
\gls{DWARF} debugging information instead. The goal is to provide an API which
is proven to be suitable for Lua FFIs, and at the same time not force
programmers who have used the LuaJIT FFI —or the standalone \verb|luaffi|— to
learn how to use a completely different API.

\subsection{The \texttt{eol} Namespace}

Where the LuaJIT FFI and \verb|luaffi| modules provide their functions in the
\verb|ffi| namespace, \Eol* provides its functionality in the \verb|eol|
namespace. Also, loading the \verb|eol| module should be possible using the
standard Lua module loader, via the \Mlua|require()| function:

\begin{luacode}
eol = require("eol")
\end{luacode}

Unless started otherwise, in the specification of the functions summarized in
\autoref{tab:eol-api-functions-summary} parameters named \texttt{typevalue}
accept both a \textsf{CType} values, or \textsf{Variable} values, in which
case the type associated with the variable will be used. Also, functions can
generate Lua errors when the types of values passed to them are incorrect.


\begin{table}[ht]
	\centering
	\begin{tabular}{lcc}
		\toprule
		Function & LJ & Err \\
		\midrule
\Mlua|library   = eol.load(name, global)|  & \Tick & \Tick \\
\Mlua|typeinfo  = eol.type(library, name)| &       & \Tick \\
\Mlua|typeinfo  = eol.typeof(typevalue)|   & \Tick & \Tick \\
\Mlua|size      = eol.sizeof(typevalue)|   & \Tick &       \\
\Mlua|alignment = eol.alignof(typevalue)|  & \Tick &       \\
\Mlua|offset    = eol.offsetof(typevalue, field)| & \Tick &\\
\Mlua|value     = eol.cast(typeinfo, value)| & \Tick & \\
\Mlua|flag      = eol.abi(parameter)| & \Tick & \\
\bottomrule
\end{tabular}

	\vspace{2pt}

	\begin{small}
	\begin{tabular}{lp{0.65\textwidth}}
		\emph{LJ} & \emph{Function available in the LuaJIT FFI module} \\
		\emph{Err}& \emph{May raise a Lua error while reading debugging information} \\
	\end{tabular}
	\end{small}

	\caption{API functions in the \texttt{eol} namespace}
	\label{tab:eol-api-functions-summary}
\end{table}


% eol.cdef(text)
% 	UNAVAILABLE / UNNEEDED

% eol.C
% 	libc access (UNIMPLEMENTED)

\subsubsection{Function \texttt{eol.load}}
	\label{sec:eol-api-load}

\begin{luacode}
  library = eol.load(name, global)
\end{luacode}

Loads a library given its \texttt{name}, and returns it as a \textsf{Library}
userdata value. The library \texttt{name} is specified without the file
extension, because the module will add the appropriate extension for the
operating system being used (e.g. \texttt{.so} for GNU/Linux). The library is
then searched for the given order of preference:

\begin{enumerate}

	\item \texttt{name} is an absolute path, and points to an existing file

	\item \texttt{name} is a relative path which, using the working directory as
	starting point, can be resolved to an existing file

	\item \texttt{name} does not contain path separators, and a library with
	a matching name exists in one of the standard locations for shared libraries
	of the operating system being used (e.g. \texttt{/lib}, \texttt{/usr/lib},
	and \texttt{/usr/local/lib} for most Unix-like systems, including GNU/Linux)

\end{enumerate}

If a suitable library could not be found using the method outlined above, or
it could not be lodeaded, the function raises a Lua error.

The \texttt{global} parameter is a boolean value which determines how the
symbols from the loaded library interact with the ones from other libraries.
When \Mlua|true|, the symbols defined by the library will be made available
for symbols resolution of subsequently loaded libraries. The parameter is
optional, and if not supplied the option is disabled as if \Mlua|false| was
supplied as the second paramter. In an Unix-like system, this is equivalent to
using \texttt{RTLD\_GLOBAL}, and \texttt{RTLD\_LOCAL} when
\texttt{dlopen()}~\cite{opengroup-dlopen} is used to load a shared object file,
respectively.

\subsubsection{Function \texttt{eol.type}}
	\label{sec:eol-api-type}

\begin{luacode}
typeinfo = eol.type(library, name)
\end{luacode}

Obtains the information for a type of a given \texttt{name},
contained in a \texttt{library}. The result is returned as a \textsf{CType}
userdata. If the type is not found in the library, \Mlua|nil| is returned
instead.


\subsubsection{Function \texttt{eol.typeof}}
	\label{sec:eol-api-typeof}

\begin{luacode}
	typeinfo = eol.typeof(typevalue)
\end{luacode}

Obtains the type of \texttt{typevalue}, and returns it as a \textsf{CType}
userdata.

\begin{luacode}
	typeinfo = eol.typeof(name)
\end{luacode}

Alternatively, it is possible to pass a string with the \texttt{name} of
a type, and it will be searched for in all the currently loaded libraries.
This second invocation method is an \Eol* extension which is not available in
the LuaJIT FFI module.

\subsubsection{Function \texttt{eol.sizeof}}

\begin{luacode}
  size = eol.sizeof(typevalue)
\end{luacode}

Obtains the size of \texttt{typevalue}, in bytes. If \texttt{typevalue} is
a \textsf{CType} userdata, the size returned corresponds to the size of values
of the type. If the size is not known (e.g. for \Mc|void|, or functions),
\Mlua|nil| is returned instead.

For any other values, an error is raised.

\subsubsection{Function \texttt{eol.alignof}}
	\label{sec:eol-api-alignof}

\begin{luacode}
	alignment = eol.alignof(typevalue)
\end{luacode}

Obtains the minimum required alignment for \texttt{typevalue}, in bytes.


\subsubsection{Function \texttt{eol.offsetof}}
	\label{sec:eol-api-offsetof}

\begin{luacode}
offset = eol.offsetof(typevalue, field)
\end{luacode}

Obtains the offset in bytes of \texttt{field} inside \texttt{typevalue}, which
must be a record data type (a \Mc|struct| in C). The \texttt{field} can be
specified as positive integer, or as a string. In the latter case, if there is
no field with the given name, a Lua error is raised.

\subsubsection{Function \texttt{eol.cast}}
	\label{sec:eol-api-cast}

\begin{luacode}
	value = eol.cast(typeinfo, value)
\end{luacode}

Creates and returns a new \textsf{Variable} userdata which describes the same
memory area as the passed \texttt{value}, but associates a new
\texttt{typeinfo} to it.

This function can be used to change the type associated with a \textsf{Variable}
userdata, without changing the value itself. It is useful to manually override
the pointer compatibility checks, or to convert between pointer values and
addresses represented as integers.

% ctype = eol.metatype(ct, metatable)
%
% cdata = eol.gc(cdata, finalizer)
%
%
%
% status = eol.istype(ct, obj)
%
% eol.copy(dst, src, len)
% eol.copy(dst, str)
% eol.fill(dst, len, [, c])


\subsubsection{Function \texttt{eol.abi}}
	\label{sec:eol-api-abi}

\begin{luacode}
flag = eol.abi(param)
\end{luacode}

Returns \Mlua|true| if \texttt{param} (a Lua string) applies for the target
\gls{ABI}. Returns \Mlua|false| otherwise. The defined parameters are detailed
in \autoref{tab:eol-abi-params}. This function is provided for compatibility
with the LuaJIT FFI module.

\begin{table}[htH]
	\centering
	\begin{tabular}{ll}
		\toprule
		Parameter & Description \\
		\midrule
		\texttt{"32bit"} & The architecture uses 32-bit wide words. \\
		\texttt{"64bit"} & The architecture uses 64-bit wide words. \\
		\midrule
		\texttt{"le"} & Little-endian architecture. \\
		\texttt{"be"} & Big-endian architecture. \\
		\bottomrule
	\end{tabular}
	\caption{Defined parameters for \texttt{eol.abi()}}
	\label{tab:eol-abi-params}
\end{table}


% \subsubsection{Variable \texttt{eol.os}}
%
% \begin{luacode}
% operatingsystem = eol.os
% \end{luacode}
%
%
% \subsubsection{Variable \texttt{eol.arch}}
%
% \begin{luacode}
% architecture = eol.arch
% \end{luacode}


\subsection{Library userdata}
	\label{sec:eol-api-library-t}

\begin{luacode}
libc = eol.load("libc")
stdout = libc.stdout  -- Obtain a variable
libc.fputs(stdout, "Hello, libc\n")  -- Obtain a function
\end{luacode}

Userdata values of type \textsf{Library} represent a loaded library. The only
way of obtaining them is using the \texttt{eol.load()} function
(\autoref{sec:eol-api-load}). Indexing a library with a string key looks up
the symbol of the same name, with one of the following results:

\begin{itemize}

	\item If the symbol refers to executable code, a \textsf{Function} userdata
	(\autoref{sec:eol-api-function-t}) is returned.

	\item If the symbol refers to data, a \textsf{Variable} userdata
	(\autoref{sec:eol-api-variable-t}) is returned.

	\item Otherwise, \Mlua|nil| is returned.

\end{itemize}

Note that it is not possible to obtain a \textsf{CType} userdata directly from
a library. The \texttt{eol.type()} function (\autoref{sec:eol-api-type}) must
be used to that effect.


\subsection{CType Userdata}
	\label{sec:eol-api-ctype-t}

Userdata values of type \textsf{CType} represent information about types used
by the native code of libraries. There are three ways in which values can be
obtained:

\begin{itemize}

	\item Using the \texttt{\_\_type} key to index a \textsf{Variable}
	userdata (\autoref{sec:eol-api-variable-t}).

	\item Using the \texttt{eol.typeof()} function
	(\autoref{sec:eol-api-typeof}).

	\item Using the \texttt{eol.type()} function (\autoref{sec:eol-api-type}).

\end{itemize}


\subsubsection{Value Construction}

\begin{luacode}
new_value = typeinfo(n)
\end{luacode}

A \textsf{CType} userdata is also a \gls{constructor} for values of the type
it represents, by means of its \texttt{\_\_call} metamethod. Invoking
a \value{CType} as a constructor accepts an optional parameter: if supplied,
an array of \texttt{n} elements is created; otherwise a single element is
created. The memory used to store the value is initialized by filling it with
zeroes (\texttt{0x00}).

All the values created this way are subject to \gls{GC}, as specified in
\autoref{sec:design-gc-interaction}.


\subsubsection{Type information}

Information about the represented data type can be obtained by indexing
\textsf{CType} values (by means of an \texttt{\_\_index} metamethod) with the
keys detailed in \autoref{tab:eol-api-userdata-keys}.

\begin{table}[ht]
	\centering
	\begin{tabular}{lp{0.7\textwidth}}
		\toprule
		Key & Description \\
		\midrule
		\texttt{name} & Name of the type, as a string. \\
		\texttt{sizeof} & Size of values of the type, in bytes. Equivalent to
			calling \texttt{eol.typeof()} passing the \textsf{CType} userdata as
			a parameter. \\
		\texttt{readonly} & Boolean value; indicates whether the type is
			declared as readonly (e.g. using \Mc|const| in C). \\
		\texttt{kind} & String which represents the kind of type, e.g.
			\texttt{"struct"}, \texttt{"union"}... \\
		\texttt{type} & For types which are defined in terms of another
			\emph{base type}, the \textsf{CType} userdata for the base type.
			Otherwise \Mlua|nil|. \\
		\bottomrule
	\end{tabular}
	\caption{Keys available in \textsf{CType} userdata}
	\label{tab:eol-api-userdata-keys}
\end{table}

For compound data types (in C, \Mc|struct|s and \Mc|union|s), two additional
operations are supported on their \textsf{CType} userdata. The Lua length
operator (\texttt\#, by means of a \texttt{\_\_len} metamethod) returns the
number of members in the compound type, and indexing the userdata as an array
—using numeric indexes— returns information about its \emph{nth} member, as
a Lua table which contains the fields specified in
\autoref{tab:eol-api-ctype-compound-member-fields}.

\begin{table}[ht]
	\centering
	\begin{tabular}{lccp{0.6\textwidth}}
		\toprule
		Key & Enum & Struct & Description\\
		\midrule
		\texttt{name}  & \Tick & \Tick & Name of the member, as a string. \\
		\texttt{value} & \Tick &       & Value, as an integer. \\
		\texttt{type}  &       & \Tick & Type of the member, as a \textsf{CType} userdata. \\
		\texttt{offset}&       & \Tick & Offset of the member, in bytes.
			Equivalent to calling \texttt{eol.offsetof()} passing the the
			\textsf{CType} userdata and the member name as paramters. \\
		\bottomrule
	\end{tabular}
	\caption{Keys available in compound \textsf{CType} member information.}
	\label{tab:eol-api-ctype-compound-member-fields}
\end{table}


\subsubsection{Method \texttt{:pointerto()}}

\begin{luacode}
pointer_typeinfo = typeinfo:pointerto()
\end{luacode}

Uses \texttt{typeinfo} as base type to construct a new \textsf{CType} userdata
value which represents a pointer to a value of the base type.


\subsubsection{Method \texttt{:arrayof(n)}}

\begin{luacode}
array_typeinfo = typeinfo:arrayof(n)
\end{luacode}

Uses \texttt{typeinfo} as base type to construct a new \textsf{CType} userdata
value which represents an array of \texttt{n} elements of the base type.


\subsection{Function userdata}
	\label{sec:eol-api-function-t}

Userdata values of type \textsf{Function} represent any piece of native code
from a \textsf{Library} which can be invoked transparently from Lua.

Information about a \textsf{Function} can be obtained by indexing the userdata
(by means of an \texttt{\_\_index} metamethod) with the keys detailed in
\autoref{tab:eol-api-function-keys}.

\begin{table}[ht]
	\centering
	\begin{tabular}{lp{0.7\textwidth}}
		\toprule
		Key & Description \\
		\midrule
		\texttt{\_\_name} & Name of the function, as a string. \\
		\texttt{\_\_type} & Type of the return value, as a \textsf{CType}
			userdata, or \Mlua|nil| if the function does not return a value. \\
		\texttt{\_\_library} & Library which contains the function, as a
			\textsf{Library} userdata. \\
		\bottomrule
	\end{tabular}
	\caption{Keys available in \textsf{Function} userdata}
	\label{tab:eol-api-function-keys}
\end{table}


\subsection{Variable userdata}
	\label{sec:eol-api-variable-t}

Userdata values of type \textsf{Variable} represent native data values. Each
value has a pointer to the region of memory occupied by the actual data, and
an associated \textsf{CType} which determines how the pointer to the data is
used.

The actual value represented by the \textsf{Variable} userdata and information
about them can be obtained by indexing the userdatas (by means of an
\texttt{\_\_index} metamethod) with the keys detailed in
\autoref{tab:eol-api-variable-keys}.

For \textsf{Variable}s with an associated array \textsf{CType}, it is possible
to manipulate the variable directly as if it were a Lua array, using the
\Mlua|variable[index]| syntax, and the length operator (\texttt\#) returns
the number of elements in the array.

% \texttt{\_\_index} and \texttt{\_\_newindex} metamethods allow
%
% contents using normal array
% to use the Lua length operator (\texttt\#, by means of a \texttt{\_\_len}
% metamethod) to obtain the number of elements in the array, and manipulating
% the values of individual elements using numeric indexes (both reading and
% writing values of the elements are possible, by means of the
% \texttt{\_\_index} and \texttt{\_\_newindex} metamethods, respectively).

\begin{table}[ht]
	\centering
	\begin{tabular}{lcp{0.65\textwidth}}
		\toprule
		Key & Writable & Description \\
		\midrule
		\texttt{\_\_value} & \Tick & Value of the variable. \\
		\texttt{\_\_name} & & Name of the variable, as a string. \\
		\texttt{\_\_type} & & Type of the variable, as a \textsf{CType} userdata. \\
		\texttt{\_\_library} & & Library which contains the variable, as
			a \textsf{Library} userdata. It may be \Mlua|nil| for variables created from Lua. \\
		\bottomrule
	\end{tabular}
	\caption{Keys available in \textsf{Variable} userdata}
	\label{tab:eol-api-variable-keys}
\end{table}



\subsubsection{Type information}

\textsf{Function} userdata values provide information about their return type
when indexing them with the \texttt{\_\_type} key, as seen in the previous
section. Type information for function parameters is also provided: applying
the Lua length operator (\texttt\#, by means of a \texttt{\_\_len} metamethod)
returns the number of parameters accepted, and indexing the userdata as an
array —using numeric indexes— returns the type information for the paramters
as \textsf{CType} userdata.


\subsubsection{Invocation}

The \texttt{\_\_call} metamethod is implemented for \textsf{Function}
userdata values, effectively making them directly callable from Lua. Invoking
native code involves:

\begin{enumerate}

	\item Checking that the number of parameters passed to the function from Lua
	match the amount accepted by the native function.

	\item Allocating as much space as needed to pass the parameters to
	the native Function, plus the space needed for the return value (if any).
	The amount of space needed must be calculated using the sizes of the native
	types, as used by the native function.

	\item For each value passed as a parameter in Lua:

		\begin{enumerate}

			\item Checking that the type of the Lua value is compatible and can be
			converted to a value of the type expected by the native function.

			\item Converting the Lua value to the corresponding native type, and
				storing the result in the allocated space.

		\end{enumerate}

	\item Re-arranging the data as needed, if the in-memory layout of the
		allocated data does not match the layout defined by the \gls{ABI} of the
		target architecture and operating system.

	\item Invoking the native function by jumping to its start address.

	\item Converting the return value, if any, to a Lua value, and pushing
		and pushing it into the Lua stack.

\end{enumerate}


\beforeintro
