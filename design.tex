% vim: ft=tex spell spelllang=en ts=2 sw=2
\cleardoublepage
\chapter{Analysis \& Design}

% \section{Analyzing DWARF}
% \subsection{Debug Information Structure}
% \subsubsection{Location of the Debug Information}


\section{Design}


\subsection{Naming}

In the Lua community there is a certain tradition of naming projects after
celestial bodies, or terms related to them —after all, Lua means \emph{moon}
in Portuguese—, but unfortunately the name initially chosen for the project
was Eris —a dwarf planet, neither a planet nor a moon— was already being used
by another Lua-related project\footnote{The Eris persistence system,
\url{https://github.com/fnuecke/eris}}. A closer inspection showed that other
dwarf planet names were already in use for software projects, so in the end
it was needed to draw inspiration from a different area.

Eöl, also known as “The Dark Elf”, is a fictional character in
J. R. R. Tolkien's Middle-earth legendarium, who is said to be the elf with
closest relationships with dwarves, and one of the first able to speak their
language.  \Eol* can also be an \gls{backronym} for “ELF Object Loader”,
which describes well the purpose of the developed solution.

\tikzstyle{component} = [
  draw=black,
  thick,
  fill=green!10,
  rectangle,
  text centered,
  minimum height=2em,
  text width=6em,
  rounded corners,
  drop shadow,
]
\tikzstyle{uses} = [
  draw,
  very thick,
  >=triangle 45,
  ->,
  dashed,
]
\tikzstyle{contains} = [
  draw,
  thick,
  >=triangle 45,
  -*,
]

\begin{figure}
  \centering
  \begin{tikzpicture}[node distance=2cm]

    \node[component] (library) {Library};
    \node[component] (typecache) [above of=library]   {Type Cache};
    \node[component] (ctype)     [right=1cm of library]   {CType};
    \node[component] (function)  [right=1cm of ctype]     {Function};
    \node[component] (variable)  [right=1cm of function]  {Variable};
    \node[component] (typeinfo)  [above of=function]  {Type Information};
    \node[datain]    (dwarf)     [above=1cm of typecache]
                                  {DWARF debugging information};

    \node (luadata) [below right of=ctype] {Visible in Lua as userdata};
    \node (elf)     [above=0cm of dwarf]   {ELF shared object};

    \path[uses] (function) -- (typeinfo);
    \path[uses] (variable) -- (typeinfo);
    \path[uses] (ctype) -- (typeinfo);
    \path[uses] (typecache) -- (dwarf);
    \path[contains] (library) -- (typecache);
    \path[contains] (typecache) -- (typeinfo);

    \begin{pgfonlayer}{background}
      \node[datablob] (elfbox) [fit=(dwarf) (elf), drop shadow] {};
      \node[fill=yellow!20, rectangle, rounded corners] (wrappers)
      [fit=(library) (variable) (function) (luadata)] { };
    \end{pgfonlayer}
  \end{tikzpicture}
  \caption{Architecture of \Eol*.}
  \label{fig:eol-architecture}
\end{figure}


