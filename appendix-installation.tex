% vim: set ft=tex spelllang=en ts=2 sw=2 et

\chapter{Installation}

\section{Prerequisites}
\label{sec:eol-prereqs}

Instead of providing its own implementation for certain functionality, \Eol*
uses existing, proved software components.

\begin{table}[h]
	\centering
	\begin{tabular}{lrccc}
		\toprule
		Component & Version  & Required & Optional & Bundled \\
		\midrule
		Lua             & 5.3      & \Tick &       & \Tick \\
		LuaBitOp        & 1.0.2    &       & \Tick & \Tick \\
		\verb|libdwarf| & 20150507 & \Tick &       & \Tick \\
		\verb|libelf|   & 0.152    & \Tick &       & \\
		\verb|readline| & 5.0      &       & \Tick & \\
		\verb|libffi|   & 3.1      &       & \Tick & \\
		\bottomrule
	\end{tabular}
	\caption{Dependencies}
	\label{tab:eol-dependencies}
\end{table}

\autoref{tab:eol-dependencies} shows the dependencies expected to be
installed in the system. The items marked (\inlinesymbol\Tick) as
\emph{bundled} are not included in the source repository, but the build system
includes support for downloading tarballs with the source code and doing
a local build. When enabled, bundled dependencies will be automatically
downloaded, built, and used instead instead of the versions provided by the
system. In the case of using \verb|libdwarf| bundled, it will be statically
linked. See~\autoref{sec:running-configure} for instructions to enable
bundled libraries. This is particularly useful for systems which do not
provide Lua 5.3 packages (for example, the case Debian and Ubuntu at the time
of writing).

\begin{table}
  \begin{tabular}{ccp{0.4\textwidth}}
  \toprule
	Distribution & Installation Command & Packages \\
	\midrule
	Debian, Ubuntu &
    \verb|apt-get install| &
    \verb|libdwarf-dev| \verb|ninja-build| \\
	Arch Linux & \verb|pacman -S| & \verb|libdwarf| \verb|ninja| \verb|lua| \\
	\bottomrule
  \end{tabular}
  \caption{Dependency packages in popular GNU/Linux distributions.}
  \label{tab:distro-dependency-packages}
\end{table}

\autoref{tab:distro-dependency-packages} shows required packages as provided
by popular GNU/Linux distributions. Some versions of Debian (and derivatives
like Ubuntu) include only a static version of \verb|libdwarf| in the packages,
most likely not built as \gls{PIC}, which is a requirement.


\section{Building}

The build process follows the convention pioneered by GNU
Autotools~\cite{autotools-history}, in which an autoconfiguration script
(\verb|configure|) is run first to inspect the system, determine which
optional components are to be enabled at build time, and generate the needed
build files. In short, building \Eol* is done by executing the following
commands from the top level source directory:

\begin{minted}{sh}
	./configure
	make
\end{minted}

or, using Ninja~\cite{ninja-manual}:

\begin{minted}{sh}
	./configure
	ninja
\end{minted}


\subsection{Autoconfiguration}
  \label{sec:running-configure}

The \verb|configure| script accepts a number of command line parameters, which
determine how the system is built. In most cases the script will figure out
automatically whether the required prerequisites (\autoref{sec:eol-prereqs})
are available, and whether the bundled versions should be used. Passing
parameters to the script is useful in case the detection fails, or to force
certain build options.  The following are the parameters most commonly used
with the \verb|configure| script:

\begin{description}

	\item [\texttt{--enable-bundled-libdwarf}] \hfill\\
		Uses the bundled \verb|libdwarf| instead of trying to use the one
		provided by the system.

	\item [\texttt{--enable-bundled-lua}] \hfill\\
		Uses the bundled Lua distribution instead of trying to use the one
		provided by the system.

	\item [\texttt{--enable-ffi}] \hfill\\
		Always use \verb|libffi| to perform function calls, and do not build
		support their JIT compilation.

	\item [\texttt{--jit-arch=ARCH}] \hfill\\
		Skip detection of the operating system and processor, and use JIT
		compilation for the supplied architecture (\verb|ARCH|). It is
		possible to obtain a list of the supported architectures by running
		\texttt{./configure --jit-arch=help}.

\end{description}

In order to obtain a complete list of all the command line options that the
script accepts, use:

\begin{minted}{sh}
	./configure --help
\end{minted}


\section{Testing the Build}

Once \Eol* is has been built, it is recommended to run the test suite to
ensure that the binaries work as expected. The test suite is included in the
source tree, and it does not require any additional dependencies. In order
to run the test suite, use the \verb|tools/run-tests| script from the
top level firectory of the source tree:

\begin{minted}{sh}
	./tools/run-tests
\end{minted}
