% vim: ft=tex spell spelllang=en ts=2 sw=2

\chapter{Contextualization}
% TODO: \minitoc ?

Computing would not be understood without the accompanying tools which enable
IT professionals to actually \emph{do something} with computers: from simple
switches and lights in the early times, to the sophisticated programming
languages and tools of the present times, all of them enable humans to
\emph{instruct machines to do things}.

%% TODO

\section{Dynamic Programming Languages}

Dynamic languages~\cite{tratt-dynamic-langs} are a family of high-level
programming languages which, at runtime, execute programming behaviors that
other programming languages perform during compilation. Runtime behaviors
could include extension of the program, by adding new code, by extending
objects and definitions, or by modifying the type system. Support for these
operations is provided directly by the language. Most dynamic languages are
also dynamically typed, and are frequently called “scripting languages”.

Popular contemporary scripting languages include Perl, Python, PHP, the
ubiquitous JavaScript, and of course Lua (described in
\autoref{sec:lua-programming-lang}). Many of their features were first
implemented as native features of the Lisp programming language, which itself
is dynamic.

Dynamic programming languages have grown in popularity in the last decades.
Programmers like them for their expressiveness, which allows for higher
productivity, and the ever increasing computing power of computers makes it
feasible to use them for performance-critical tasks traditionally left for
statically compiled languages. The development of novel compilation techniques
aimed at this family of languages, and the relentless improvement of their
virtual machines (see \autoref{sec:virtual-machines}), has contributed as well
to increase widespread adoption.

\subsection{Virtual Machines as Runtime Environments}
  \label{sec:virtual-machines}

A \gls{VM} is an \gls{emulation} of a particular computer system based on its
architecture. \emph{System virtual machines} provide a complete substitute for
the real machine, and targeted towards the execution of complete operating
systems and software stacks. On the other hand, \emph{process virtual
machines} execute a single computer program by providing an execution
environment suitable for a particular programming language, which can be
platform-independent.

Process \gls{VM}s provide a so called \emph{runtime environment}. They run as
a normal application inside the host operating system. This approach has
several advantages:

\begin{itemize}
	\item Provides a platform-independent programming environment,
		abstracting away details of the underlying hardware or operating system,
		potentially allowing a program to execute in the same way on any platform.
	\item The level of abstraction provided by the \gls{VM} is higher than that
		of a low-level \gls{ISA}, which allows it to provide services rarely
		available in real machines (e.g. automatic memory management).
	\item There is an additional level of isolation between the operating system
		and the execution environment. This allows discretionary control over the
		executed code, which is confined to the bounds allowed by the \gls{VM}.
\end{itemize}

The main downside of virtual machines is the lower performance compared to the
execution of native code. Performance levels comparable to compiled languages
can be obtained using a combination of \gls{JIT} compilation (see
\autoref{sec:jit-compilers}), and compiler-based optimization techniques
specifically designed for them.

In some regards, the additional level of isolation provided by a \gls{VM}
could be a nuisance due to the fact that it could prevent programs from
accessing resources (either hardware-based, or provided by the host system)
needed for its operation. For this reason, many virtual machines provide
a \gls{FFI}: a mechanism to call-out native code and re-gain access to the
whole system, while still being under the control and supervision of the
\gls{VM}.


\subsection{JIT Compilers}
	\label{sec:jit-compilers}

% TikZ styles for box diagrams
\tikzstyle{bdBox} = [
	rectangle, drop shadow, draw=black, thick, fill=white,
	text centered, minimum height=2em, minimum width=3em,
]
\tikzstyle{bdProcBox} = [
	rounded corners, fill=blue!10, text centered
]
\tikzstyle{bdProcLine} = [
	draw, thick, color=blue!20
]
\tikzstyle{bdCircle} = [
	circle, fill=blue!20, draw=black,
]
\tikzstyle{bdLine} = [draw, thick]
\tikzstyle{bdArrow} = [bdLine, >=triangle 45, ->]

\begin{figure}[ht]
	\centering
	\begin{tikzpicture}[node distance=1.4cm]
		\small
		\node[bdBox] (srcbox) [start chain=going right, on chain] {Source code};
		\node[bdCircle] (aot) [on chain] { };
		\node[bdProcBox] (aotlabel) [below of=aot] { \emph{Basic JIT compilation} };
		\node[bdBox] (bytecode) [on chain] {Bytecode};
		\node[bdCircle] (jit) [on chain] { };
		\node[bdProcBox] (jitlabel) [below of=jit] { \emph{Optimizing JIT compilation} };
		\node[bdBox] (bincode) [on chain] {Machine code};

		\node (profileinfo) [above of=bytecode, xshift=1cm] {Runtime statistics};

		\path[bdLine] (srcbox.east) -- (aot.west);
		\path[bdArrow] (aot.east) -- (bytecode.west);
		\path[bdLine] (bytecode.east) -- (jit.west);
		\path[bdArrow] (jit.east) -- (bincode.west);
		\path[bdLine] (profileinfo) -- (jit);

		\path[bdProcLine] (aot.south) -- (aotlabel.north);
		\path[bdProcLine] (jit.south) -- (jitlabel.north);
	\end{tikzpicture} \caption{Two-tier JIT compilation}
	\label{fig:two-tier-jit} \end{figure}

\acrlong{JIT} compilation is a term which refers to any technique that
performs compilation of program code during the execution of the program —at
runtime—, instead of doing it before execution. This is also known as
\emph{dynamic translation}. Typically a VM which uses JIT compilation contains
two compilers (\autoref{fig:two-tier-jit}):

\begin{enumerate}

		\item The first compiler —known as \emph{tier 1}— performs a translation
		of the input source code into an intermediate binary representation known
		as \emph{bytecode}, which is faster to execute than interpreting the
		source code in its original form. In order to start executing the
		program as quickly as possible, this first compiler performs only basic
		optimizations—if any. Additionally, the compiler adds instrumentation code
		in the generated bytecode, which gathers statistics about program execution.
		
		\item The second compiler —known as \emph{tier 2}— generates either
		bytecode which is better optimized, or machine code, at the cost of
		slower compilation speed, compared to the tier 1 compiler. The
		statistics gathered by the instrumentation code inserted by the tier
		1 compiler are used to determine which parts of the program code are used
		more often, and only those are recompiled. Depending on the
		implementation, compilation is done per source file, per function, or even
		for arbitrary code fragments. The bytecode produced by the tier 1 compiler
		can be used as input to avoid having to re-parse the program source code
		in its textual form.

\end{enumerate}
		
Some VMs do not provide a bytecode interpreter, and must always generate
machine code. In this scenario, the tier 1 compiler also generates machine
code, but unoptimized to avoid having long delays during the startup process
of the VM when the tier 1 compiler is used.

Code generated by JIT compilers offers better performance than interpreters,
and in some cases it can perform better than statically compiled code because
a JIT compiler can use optimizations which are only feasible at runtime:

\begin{itemize}

	\item Generated code can be optimized for the exact processor and operating
	system where the application runs. This can be done with traditional
	compilers, but it requires compiling the code once for each combination of
	target processor and operating system.

	\item The VM can collect statistics about how the program is running, and
	rearrange and recompile the code according to them.

	\item The compiler can make optimistic assumptions about the program,
	generate machine code that works optimally in most of the situations, plus
	additional checks to know whether the assumptions hold, falling back to
	using the original implementation (or a different re-compilation) otherwise.
	For example, this strategy is used to inline method calls assuming that
	\gls{dynamic-dispatch} will not be needed, and performing the dynamic
	dispatch when the types of the involved objects to not match the ones assumed.

	\item Code can be rearranged after observing how it uses the cache memory to
	make a better usage of it.

\end{itemize}

The introduction of JIT compilation in VMs has allowed existing dynamic
languages to achieve levels of performance comparable to those of languages
compiled to machine code~\cite{lj-perf1}. The JavaScript programming language
is a paradigmatic example: it has existed since 1995, yet it has gained much
more widespread usage after the introduction in 2008 of JavaScript engines
capable of JIT compilation (Mozilla's TraceMonkey was the first, with Google's
V8 and WebKit's JavaScriptCode following right after~\cite{js-raceforspeed}),
making it feasible to use JavaScript as a general purpose language outside of
web browsers.


\subsection{The Lua programming language}
	\label{sec:lua-programming-lang}

The Lua language is a “powerful, fast, lightweight, embeddable scripting
language” \cite{lua-about}. It was initially created as a data description
language at \gls{PUC-Rio}, to be used for in-house software development, and
has since evolved into a general purpose programming language. It has been
used in proffessional applications (e.g. Adobe Lightroom) and it has seen
widespread usage in the video games industry (e.g. World Of Warcraft).

\begin{listing}[H]
  \begin{luacode}
    -- Create a table, with one key, "age" and 7 as value
    animal = { age = 7 }

    -- Associate a string value to the "kind" table key
    animal["kind"] = "cat"

    -- Keys which are valid identifiers can be accessed with "."
    animal.name = "Doraemon"

    -- The dot "." syntax workds for adding functions to tables
    function animal.describe(self)
      print(self.name .. " is a " .. tostring(self.age) ..
            "-year old " .. self.kind)
    end

    -- This is equivalent to: animal.describe(animal)
    animal:describe()  --> Doraemon is a 7-year old cat

    -- Adding a function with colon ":" adds an implicit "self"
    function animal:furryness()
      return self.kind == "cat" and "high" or "unknown"
    end

    animal:furryness()  --> high
  \end{luacode}
	\caption{Lua tables being used as objects}
  \label{lst:lua-example-tables}
\end{listing}

Lua's main programming paradigm is imperative, but the language supports
functions as \glspl{first-class-value} and \glspl{closure}, making it possible
to easily write programs in a functional programming style. Like in
\gls{pascal}, English words (\Mlua|function|, \Mlua|then|, \Mlua|end|) are
used as delimiters for language constructs. Another defining characteristic of
the language is that, by design, it only provides one compound data structure,
the \emph{table}, which is the basis for all user--defined types. Tables can
be used as arrays (listing \ref{lst:lua-example-memoize}), structures, and
objects (listing \ref{lst:lua-example-tables}).

A unique and powerful feature of Lua is its support for \emph{metatables}:
values may have an associated table (the so-called \emph{metatable}) which
allows to extend the semantics of language constructs, allowing to define how
tables behave when arithmetic and relational operators are applied to tables,
or how table fields are accessed. Listing \vref{lst:lua-example-memoize}
demostrates how customizing table access can be used to define a
seemingly--infinite array which contains the $n^{th}$ \gls{fibonacci-number}
at index $n$. A common use for metatables is enabling support for
object--oriented programming, using them to define object inheritance chains
(listing \ref{lst:lua-example-meta-oo}) --- as Lua itself does not have the
notion of classes, prototypes are used instead, as in the Self or JavaScript
languages.

\begin{listing}[H]
  \begin{luacode}
    fib = { 1, 1 }
    setmetatable(fib, {
      __index = function (values, n)
        -- Calculate and memoize the Fibonacci(n)
        values[n] = values[n - 1] + values[n - 2]
        return values[n]
      end
    })
    print(fib[10])  --> 55
  \end{luacode}
  \caption{\Gls{memoization} and \gls{dynamic-programming} using a Lua metatable}
	\label{lst:lua-example-memoize}
\end{listing}

\begin{listing}[H]
  \begin{luacode}
    -- Base object describing an unnamed living creature
    animal = {
      name = "Unnamed",
      kind = "living creature",
      describe = function (self)
        print(self.name .. " is a " .. self.kind)
      end,
    }

    -- When indexing the table passed as first argument, fields will
    -- be looked up from the "animal" table associated to the "__index"
    -- key of the metatable. The function returns the first argument.
    cat = setmetatable({ kind = "cat", name = "Doraemon" },
                       { __index = animal })
    dog = setmetatable({ kind = "dog", name = "Snowy", },
                       { __index = animal })

    -- The :describe() method is searched in "animal"
    cat:describe()  --> Doraemon is a cat
    dog:describe()  --> Snowy is a dog

    -- Chained key lookup can be used to make the values from the
    -- base object the default ones
    tom = setmetatable({ name = "Tom" }, { __index = animal  })
    tom:describe()  --> Tom is a living creature
  \end{luacode}
  \caption{Lua metatables used for object inheritance}
  \label{lst:lua-example-meta-oo}
\end{listing}

Lua, starting in version 5.0~\cite{lua50-impl}, uses a register-based virtual
machine. This allows for improved performance by avoiding excessive copying of
values on stack \emph{pop} and \emph{push} operations. Traditionally, most
virtual machines intended for execution of languages are stack based,
including heavyweight, enterprise-proven systems like the Java™ JVM, and
Microsoft's .NET environment.


\section{Binding Native Code to Lua}

\subsection{Lua C API}
	\label{sec:lua-c-api}

The Lua \gls{VM} exposes a C \gls{API} which, among other things, allows to
register C functions to be called by Lua code. These communicate with the VM
using a well-defined protocol (see listing \ref{lst:lua-c-api-example}
for an example).

\begin{listing}[H]
	\begin{ccode}
  int sum_and_average (lua_State *L) {
    lua_Number result = 0.0;
    int nargs = lua_gettop (L); /* number or arguments */
    for (int i = 1; i <= n; i++) {
      if (!lua_isnumber (L, i)) {
        lua_pushliteral (L, "argument is not a number");
        lua_error (L);
      }
      result += lua_tonumber (L, i);
    }
    lua_pushnumber (L, result);        /* first result */
    lua_pushnumber (L, result / n);   /* second result */
    return 2;                     /* number of results */
  }
	\end{ccode}
	\caption{C function callable from Lua}
	\label{lst:lua-c-api-example}
\end{listing}

Despite Lua providing a register-based VM, the C API uses a \emph{virtual
stack} to exchange values with C code. When a C function is called, it gets
a new stack, initially containing the arguments passed to the function, and
it must adhere to the following protocol:

\begin{itemize}
	\item C functions must be declared as returning an \Mc|int|, and accept
		a single \Mc|lua_State*| parameter. This is, their function pointer
		type is compatible with \Mc|lua_CFunction|, which Lua defines as:
		\begin{ccode}
			typedef int (*lua_Cfunction) (lua_State*);
		\end{ccode}

	\item The C function receives arguments from Lua in its call stack in
		direct order (the first argument is pushed first by the VM). The size of
		the stack —at this point the number of arguments— can be queried using
		\Mc|lua_gettop()|, and values can be obtained using the \verb|lua_to*()|
		functions.

	\item To return values back to Lua, the C function pushes them onto the
		stack in direct order (the first result is pushed first) using the
		\verb|lua_push*()| functions.

	\item The C function passes control to the VM returning the number of
		results available at the top of the stack. Any values left in the
		stack below the results are discarded.

\end{itemize}

The C API provided by Lua is comprehensive, but using it tends to produce
verbose programs, with varying amounts of repeated code. Lua itself
acknowledges this issue by officially including an \emph{auxiliar library}
as part of the package, which provides a collection of utility functions
implemented using the base API. Third-party wrappers over the official API
exist, which can either provide a more convenient interface to Lua (like
LuaAutoC\footnote{\url{https://github.com/orangeduck/LuaAutoC}}, and
\verb|luapi|\footnote{\url{http://lua-users.org/files/wiki_insecure/users/luapi/luapi5-1.txt}}),
or allow languages other than C to use Lua (more than
20 at the time of writing, including support for many popular languages like
C++, Objective-C, Go, Java, or Fortran).


\subsection{Binding Generators}
	\label{sec:binding-generators}

Binding generators are tools that can be used to create a binding to a library
in an automated way. Often they fall into the category of \glspl{transpiler}:
they take as input the source code of the code to generate a binding for,
and generate a new set of source files which contain the code of the binding.
This set of source files are themselves compiled into a loadable module for
the target programming language or virtual machine, making it a
\emph{build-time} solution.

More than often, binding generators do \emph{not} include a full parser for
the programming language of origin, and they require to be fed a simplified
version of the code being wrapped. This can be a nuisance for code bases which
use complex language constructs unsupported by the binding generator.

A popular, general-purpose binding generator is
SWIG\footnote{\url{http://swig.org}} (Simple Wrapper and Interface Generator),
which supports creating bindings for multiple programming languages, with Lua
being just one more of the supported targets. SWIG uses its own C/C++ parser,
which, while being complete~\cite{swig3doc}, still fails on certain inputs.

There are also Lua-specific binding generators, of which the
\gls{lua-users-wiki} provides a comprehensive
list~\cite{lusers-BindingCodeToLua}. The oldest is
ToLua\footnote{\url{http://www.tecgraf.puc-rio.br/~celes/tolua/}}, maintained
by staff from \gls{PUC-Rio} like Lua itself. Being Lua the only target
language, bindings generated by ToLua —and derivatives like ToLua++— are more
idiomatic that those generated by e.g. SWIG. As a downside, ToLua's parser is
quite limited, and only recognizes a subset of C and C++, to the point that it
is advised to provide a \emph{cleaned header file} containing only
declarations of data types, functions, and C++ classes recognizable by its
parser.


\subsection{Foreign Function Interfaces}
	\label{sec:ffis}

In its most generic meaning, a \gls{FFI} is any mechanism which allows
a program written in a programming language to call code or use services
written in another. In the Lua community, a FFI refers specifically to such
a mechanism which works at run time. That is, it \emph{does not} require using
the Lua C API (\autoref{sec:lua-c-api}), it is not a binding generator used at
build time (\autoref{sec:binding-generators}), nor is it a wrapper over the
Lua C API.

FFIs can be separated in two categories, depending on how they obtain the type
information (functions names, arguments, return values; and the data types of
all the values involved):

\begin{itemize}

	\item FFIs which require the programmer to supply type information.

	\item FFIs which obtain type information in an automated fashion.

\end{itemize}

The canonical example of a FFI which requires the programmer to enter type
information is the \gls{LuaJIT} FFI
module\footnote{\url{http://luajit.org/ext_ffi.html}}. Its functionality is
available as a regular Lua module which includes a number of support
functions, including the ability to parse C-style declarations to obtain type
information. Under the hood, the module generates the machine code for wrapper
functions —of type \Mc|lua_CFunction| (c.f. \autoref{sec:lua-c-api})— using
the LuaJIT code generator, which are directly callable (example in listing
\autoref{lst:luajit-ffi-example}). A standalone
\verb|luaffi|\footnote{\url{https://github.com/jmckaskill/luaffi}} module for
the standard Lua \gls{VM} exists, which has the same interface as the LuaJIT
FFI module, and even reuses its code generator.

\begin{listing}[H]
	\begin{luacode}
  local ffi = require("ffi")
  ffi.cdef("int printf(const char *fmt, ...);")
  ffi.C.printf("Hello %s!\n", "world")
	\end{luacode}
	\caption{Using a C function with the LuaJIT FFI}
	\label{lst:luajit-ffi-example}
\end{listing}

Where the LuaJIT FFI requires the programmer to enter C-style declarations of
functions, LGI\footnote{\url{https://github.com/pavouk/lgi/}} uses the type
information supplied by GObject Introspection~\cite{gobject-introspection},
which provides access to most of the libraries included as part of the
GNOME\footnote{\url{http://gnome.org}} desktop environment. The type
information supplied by GObject Introspection is extracted from the source
code of the GNOME destop components at compile-time, and stored on disk in its
own file format. The included \verb|libgirepository| library is then used by
LGI to allow the enumeration of the available modules and their contents, as
well as the transparent invocation of functions from Lua. The metadata
recorded by GObject Introspection when scanning source code includes
additional information in specially formatted comments, which allows LGI to
create idiomatic object oriented interfaces (c.f.
listing~\autoref{lst:lua-lgi-example}).

\begin{listing}[H]
	\begin{luacode}
	local lgi = require("lgi")
	local Gtk = lgi.Gtk

	local w = Gtk.Window {
		title = "LGI Example",
		child = Gtk.Button { label = "Close",
                         on_clicked = Gtk.main_quit },
	}
	w:show_all()
	Gtk.main()
	\end{luacode}
	\caption{Using the GTK+ user interface toolkit via LGI and GObject
	Introspection}
	\label{lst:lua-lgi-example}
\end{listing}

\section{Executable Formats}

\subsection{ELF}
  \label{sec:elf}

The \acrlong{ELF} (ELF, formerly called \emph{Extensible Linking
Format}) is a common standard file format for executable programs, object
code, shared libraries, and even core dumps. Since its publication as part of
the System V Release 4 (SVR4) Application Binary Interface (ABI) specification
\cite[c.~4]{elfspec-sysv}
it has been adopted by many Unix-like (Solaris, most of the BSD variants,
GNU/Linux), and non-Unix operating systems (most notably, OpenVMS, BeOS, and
its successor Haiku).

\noindent An ELF file contains three mandatory parts
(\autoref{fig:elf-structure}):

\begin{itemize}

	\item The ELF header, which contains general information about the file
	(architecture, endianness, ABI, etc.), the offsets of the \emph{program
	header table}, and the \emph{section header table}; and their sizes.

	\item The \emph{program header table}, which describes zero or more
	\emph{segments}. Though it is not required, it is typically located right
	after the ELF header.

	\item The \emph{section header table}, which describes zero or more
	\emph{sections}.

\end{itemize}

\begin{figure}
	\centering
	\begin{tikzpicture}[node distance=1.5mm, bend angle=0]
		\small
		\node[bdBox] (elfheader) [minimum width=10em, start chain=going below, on chain] {ELF header};
		\node[bdBox] (prgheader) [minimum width=10em, on chain] {Program header};
		\node[bdBox] (sect-text) [minimum width=10em, on chain] {\verb|.text|};
		\node[bdBox] (sect-rodata) [minimum width=10em, on chain] {\verb|.rodata|};
		\node (ellipsis) [minimum width=10em, on chain, yshift=-1em] {...};
		\node[bdBox] (sect-data) [minimum width=10em, on chain, yshift=-1em] {\verb|.data|};
		\node[bdBox] (secthdrtable) [minimum width=10em, on chain] {Section header table};

		\draw[decorate, decoration={brace}] let \p1=(sect-text.north),
			\p2=(sect-rodata.south) in ($(2.2, \y1)$) -- ($(2.2, \y2)$)
			node[midway] (g1) {};

		\draw[decorate, decoration={brace}] let \p1=(ellipsis.north),
			\p2=(sect-data.south) in ($(2.2, \y1)$) -- ($(2.2, \y2)$)
			node[midway] (g2) {};

		\draw[->, bend right, >=latex, bend right, thick]
			(elfheader.east) to [out=90,in=90] (g1.east);
		\draw[->, bend right, >=latex, bend right, thick]
			(elfheader.east) to [out=90,in=90] (g2.east);

		\draw[->, bend left, >=latex, bend right, thick]
			(secthdrtable.west) to [out=90,in=90] (sect-text.west);
		\draw[->, bend left, >=latex, bend right, thick]
			(secthdrtable.west) to [out=90,in=90] (sect-rodata.west);
		\draw[->, bend left, >=latex, bend right, thick]
			(secthdrtable.west) to [out=90,in=90] (sect-data.west);
	\end{tikzpicture}
	\caption{Structure of an ELF object file}
	\label{fig:elf-structure}
\end{figure}

Consequently, with the aforementioned parts of an ELF file, the actual data is
categorized in \emph{segments}, and \emph{sections}. Segments contain
information used at runtime for the execution of the code, and sections
contain data about the code itself. Section names and contents are arbitrary,
and they are used to store information used \emph{offline} (i.e. not at
runtime) needed for relocating and linking the program code contained in the
ELF file.

Entries for both program and section header tables contain the sizes of the
segments and sections they describe, and their offsets inside the ELF file. It
is allowed by the specification to define more than one segment and section
entries which refer to the same portion of the file, and also that there are
portions of the file which do not belong to any segment or section.


\subsection{DWARF}

% The DWARF debugging information format has been designed along with
% ELF\footnote{And that is why both have names related to the medieval fantasy
% canon.}, although it is possible to store DWARF-encoded data in other object
% file formats. The DWARF specification~\cite{dwarfspecv4} defines how to encode
% all kinds of information potentially useful for a debugger, and how it must be
% laid out in different object code file formats.
%
% Each piece of information encoded in DWARF is stored in \gls{DIE}. Each DIE is
% a data records with a \emph{tag} which determines the type of information
% represented, and a payload which varies depending on the value of the tag.
% When a DIE is used to describe a data type, it is often referred to as
% a \gls{TUE}.
%
% DIEs are the  In the case of ELF files (\autoref{sec:elf}), the DWARF
% information is laid out inside the ELF file in a number of sections
%
% \begin{itemize}
%
% 	\item The \verb|.debug_info| section contains 
%	
% 	and \verb|.debug_types| sections contain a number of data records which
% 	describe the different entities present in the program, and their data
% 	types. Each one of those records is called a \gls{DIE}.
%
% \end{itemize}

The \gls{DWARF} format was first developed by the Bell Labs for the
\verb|sdb| debugger of System V Unix. Nowadays the formal specification
is available under the GNU \gls{FDL}, and its new versions have been
discussed using public channels of communication, following a
community-oriented process. Downloadable copies of all the version of
the specification are available at {\small\url{http://dwarfstd.org}}.

Debugging information is not strictly needed at runtime for the execution of
the program: it is considered ancillary, and therefore is stored in sections.
As per the specification, the DWARF debugging information must be stored in
the ELF objects in sections with the \verb|.debug_| prefix. Each one of those
sections stores a particular kind of information, as seen on
\autoref{tab:elf-dwarf-sections}.

\begin{table}
  \centering\small
  \begin{tabular}{lp{0.65\textwidth}}
    \toprule
    Section & Contents \\
    \midrule
    \verb|.debug_types| & Stores DIEs for types. \\
    \verb|.debug_info| & Stores DIEs for executable program code (functions, mostly) \\
    \verb|.debug_line| & Maps of object code positions to source code line numbers \\
    \verb|.debug_pubnames| & Public function names and entry offsets in \verb|.debug_info| \\
    \verb|.debug_pubtypes| & Public type names and entry offsets in \verb|.debug_types| \\
    \verb|.debug_str| & String data (e.g. type, variable and function names) \\
    \bottomrule
  \end{tabular}
  \caption{Main ELF sections used to store DWARF debugging information}
  \label{tab:elf-dwarf-sections}
\end{table}


In \gls{DWARF}, all the data provided by debugging information is stored in
a hierarchical tree-like structure, and each node of the tree is called
a \gls{DIE}. Each DIE consists of an identifying tag, and a series of
attributes. The tag specifies the class to which an entry belongs, and the
attributes define the characteristics of the entry. An entry, or a group of
entries together, provide a description of an entity in the corresponding
source code. The entries are contained in the \verb|.debug_info| and
\verb|.debug_types| sections of ELF object files.

%%%
%%% TikZ styles for DIE diagrams
%%%
\pgfdeclarelayer{background}
\pgfdeclarelayer{foreground}
\pgfsetlayers{background,main,foreground}

\tikzstyle{datablob} = [
  rectangle, rounded corners, drop shadow, draw=black, thick,
  text centered, minimum height=2em, minimum width=3em, fill=blue!20,
]
\tikzstyle{die}      = [start chain=going below, node distance=1mm]
\tikzstyle{dielabel} = [on chain]
\tikzstyle{dieitems} = [
  rectangle split, rectangle split parts=#1, rectangle split part align=left,
  thick, draw, fill=blue!10, on chain,
]
\tikzstyle{enumitems} = [
  rectangle split, rectangle split parts=#1, rectangle split part align=left,
  thick, rounded corners,
  color=black!50, fill=black!5, draw=black!50,
  minimum height=3em,
]
\tikzstyle{valuefrom} = [draw=black!50, thick, dashed]
\tikzstyle{arrow} = [draw, thick, >=triangle 45, ->]
\tikzstyle{datain} = [
  draw=black!80, thick, fill=blue!10, rectangle,
  text centered, minimum height=1.3em, text width=10em,
]
%%%
%%% End TikZ styles for DIE diagrams
%%%

\begin{figure}
  \centering
  \begin{tikzpicture}[die]
    \node[dielabel] (taglabel) {Tag};
    \node[dieitems=1] (tag) {\verb|DW_TAG_subprogram|};

    \node[dielabel] (attrlabel) {Attributes};
    \node[dieitems=3] (attributes) {
        \verb|DW_AT_type|
      \nodepart{two}
        \verb|DW_AT_sibling|
      \nodepart{three}
        \verb|DW_AT_…|
    };

    \node[datablob] (refdie2) [right=4cm of attributes] { };
    \node[datablob] (refdie1) [above=1mm of refdie2] { };
    \node  (refdieellipsis)   [below=1mm of refdie2] { ... };
    \node[datablob] (refdieN) [below=1mm of refdieellipsis] { };

    \node (refdieslabel) [above=3mm of refdie1] {DIEs referenced by offset};

    \path[arrow] (attributes.text east) -- (refdie1.west);
    \path[arrow] (attributes.two east) -- (refdie2.west);
    \path[arrow] (attributes.three east) -- (refdieN.west);

    \begin{pgfonlayer}{background}
      \node[datablob] (die) [fit=(taglabel) (attributes) (tag)] { };
      \node[fill=yellow!20, rectangle, rounded corners] (refdiesbg)
        [fit=(refdieslabel) (refdieN)] { };
    \end{pgfonlayer}

    \node (dielabel) [below=3mm of die] {DIE};
  \end{tikzpicture}

  \caption{DIE attribute references.}
  \label{fig:die-attr-references}
\end{figure}

DIE attributes can contain references to other DIEs, as shown in
\autoref{fig:die-attr-references}, making it possible to create arbitrary
links between nodes of the tree. Those explicit links are used extensively to
group related entries together (for example entries with tag
\verb|DW_TAG_formal_parameter|, which describe the parameters of a function,
are chained using \verb|DW_AT_sibling| attributes), and to perform
\gls{data-deduplication} of the debug information (for example, instead of
making a new entry for the \mintinline{c}{int} type, only one is created and
then referenced from other entries).


\label{sec:debuginfo-structure}

We now provide a simplified overview of the tree structure formed by the debug
information entries of a program, in a level of detail which is enough for the
goals of this project. For complete, detailed information, we refer to the
DWARFv4 Specification~\cite{dwarfspecv4}.


\subsubsection{Types}

A type is represented as a DIE with one of the following tags:

\begin{itemize}
  \item \verb|DW_TAG_base_type|
  \item \verb|DW_TAG_pointer_type|
  \item \verb|DW_TAG_typedef|
  \item \verb|DW_TAG_const_type|
  \item \verb|DW_TAG_array_type|
  \item \verb|DW_TAG_structure_type|
  \item \verb|DW_TAG_union_type|
  \item \verb|DW_TAG_enumeration_type|
\end{itemize}


\minisec{Base Types}

\begin{figure}
  \centering
  \begin{tikzpicture}[die]
    \node[dielabel] (taglabel) {Tag};
    \node[dieitems=1] (tag) { \verb|DW_TAG_base_type| };
    \node[dielabel] (attrlabel) {Attributes};
    \node[dieitems=2] (attributes) {
        \verb|DW_AT_byte_size|
      \nodepart{second}
        \verb|DW_AT_encoding|
    };

    \node[enumitems=1] (sizevalues) [right=4cm of tag] {\emph{size}};
    \node[enumitems=4] (encodingvalues) [below=1cm of sizevalues] {
        \verb|DW_ATE_signed|
      \nodepart{second}
        \verb|DW_ATE_unsigned|
      \nodepart{third}
        \verb|DW_ATE_float|
      \nodepart{fourth}
        \verb|DW_ATE_boolean|
    };

    \path[valuefrom] (attributes.text east) -- (sizevalues.west);
    \path[valuefrom] (attributes.second east) -- (encodingvalues.west);

    \begin{pgfonlayer}{background}
      \node[datablob] [fit=(taglabel) (attributes) (tag)] {};
    \end{pgfonlayer}
  \end{tikzpicture}
  \caption{DIE describing a base type.}
  \label{fig:die-base-type}
\end{figure}


Base types(\autoref{fig:die-base-type}) are represented by DIEs with
a \verb|DW_TAG_base_type| tag. This covers all the C/C++ numeric types: signed
and unsigned integers, floating point types (\mintinline{c}{float},
\mintinline{c}{double}), the \mintinline{c}{char} type, and the new
integer-based types of C99 (\mintinline{c}{_Bool}, \mintinline{c}{int32_t},
etc.)

The size of the values is given using a \verb|DW_AT_byte_size| attribute,
in bytes. Sizes are the same reported by the C \mintinline{c}{sizeof}
operator. A \verb|DW_AT_encoding| attribute specifies how the values are
used in the program. For example, \verb|DW_ATE_boolean|
corresponds with the \Mc|bool| type (or \Mc|_Bool|) in the
source program. \autoref{tab:dwarf-base-types-mapping} summarizes how C
types are represented by a base type DIE for the x86\_64 architecture
(sizes may vary in other architectures).

\begin{table}[ht]
	\begin{center}
		\begin{tabular}{lcl}
			\toprule
			C Type & \verb|DW_AT_byte_size| & \verb|DW_AT_encoding| \\
			\midrule
			\mintinline{c}{bool}               & 1 & \verb|DW_ATE_boolean| \\
			\mintinline{c}{char}               & 1 & \verb|DW_ATE_signed_char| \\
			\mintinline{c}{unsigned char}      & 1 & \verb|DW_ATE_unsigned_char| \\
			\mintinline{c}{short int}          & 2 & \verb|DW_ATE_signed| \\
			\mintinline{c}{unsigned short int} & 2 & \verb|DW_ATE_unsigned1| \\
			\mintinline{c}{int}                & 4 & \verb|DW_ATE_signed| \\
			\mintinline{c}{unsigned int}       & 4 & \verb|DW_ATE_unsigned| \\
			\mintinline{c}{long int}           & 8 & \verb|DW_ATE_signed| \\
			\mintinline{c}{unsigned long int}  & 8 & \verb|DW_ATE_unsigned| \\
			\mintinline{c}{float}              & 4 & \verb|DW_ATE_float| \\
			\mintinline{c}{double}             & 8 & \verb|DW_ATE_float| \\
			\mintinline{c}{long double}        & 16& \verb|DW_ATE_float| \\
			\bottomrule
		\end{tabular}
	\end{center}
	\caption{DWARF representation of C base types for the x86\_64 architecture}
	\label{tab:dwarf-base-types-mapping}
\end{table}

The rest of C/C++ base types are defined as aliases of these
using \mintinline{c}{typedef}.


\minisec{Pointer Types}

\begin{figure}
  \centering
  \begin{tikzpicture}[die]
    \node[dielabel] (taglabel) {Tag};
    \node[dieitems=1] (tag) {\verb|DW_TAG_pointer|};
    \node[dielabel] (attrlabel) {Attributes};
    \node[dieitems=1] (attributes) {\verb|DW_AT_type|};

    \node[dielabel] (rtaglabel) [right=3cm of tag, yshift=-3.5mm] {Tag};
    \node[dieitems=1] (rtag) {\verb|DW_TAG_pointer|};
    \node[dielabel] (rattrlabel) {Attributes};
    \node[dieitems=1] (rattributes) {\verb|DW_AT_type|};

    \node[datablob] (basedie) [right=2cm of rattributes] {Type DIE};

    \path[arrow] (rattributes.text east) -- (basedie);

    \begin{pgfonlayer}{background}
      \node[datablob] (die) [fit=(taglabel) (attributes) (tag)] {};
      \node[datablob] (rdie) [fit=(rtaglabel) (rattributes) (rtag)] {};
    \end{pgfonlayer}

    \path[arrow] (attributes.text east) -- (rdie.west);
  \end{tikzpicture}
  \caption{DIEs describing a Pointer-to-pointer-to type.}
  \label{fig:die-pointer-type}
\end{figure}

Pointer types (\autoref{fig:die-pointer-type}) are represented by DIEs with
a \verb|DW_TAG_pointer_type| tag, and a lone \verb|DW_AT_type| attribute
points to the DIE of the pointed-to type. This allows to represent pointers
of/to any type. Multiple levels of indirection are represented by chains of
\verb|DW_TAG_pointer_type| DIEs. The example in \autoref{fig:die-pointer-type}
exemplifies this situation, in what could be the representation of the
\mintinline{c}{int**} when the rightmost “Type DIE” contains the information
for the \mintinline{c}{int} type.


\minisec{Constant Types}

Flagging values of a type as constant is done in the same way as representing
a pointer type: a DIE with a \verb|DW_TAG_const_type| contains
a \verb|DW_AT_type| attribute with reference pointing to a type DIE.
This way, the pointed type is marked as immutable.


\minisec{Array Types}

\begin{figure}
  \centering
  \begin{tikzpicture}[die]
    \node[dielabel] (taglabel) {Tag};
    \node[dieitems=1] (tag) {\verb|DW_TAG_array_type|};
    \node[dielabel] (attrlabel) {Attributes};
    \node[dieitems=2] (attributes) {
        \verb|DW_AT_type|
      \nodepart{two}
        \verb|DW_AT_sibling|
    };

    \node[datablob] (basedie) [right=4cm of attrlabel] {Type DIE};

    \node[dielabel] (rtaglabel) [below=1cm of basedie] {Tag};
    \node[dieitems=1] (rtag) {\verb|DW_TAG_subrange_type|};
    \node[dielabel] (rattrlabel) {Attributes};
    \node[dieitems=2] (rattributes) {
        \verb|DW_AT_count|
      \nodepart{two}
        \verb|DW_AT_byte_size|
    };

    \node[enumitems=1] (countvalue) [right=3cm of rattributes, yshift=0.8cm] {\emph{count}};
    \node[enumitems=1] (countsize) [below=0.5cm of countvalue] {\emph{size}};

    \begin{pgfonlayer}{background}
      \node[datablob] (die) [fit=(taglabel) (attributes) (tag)] {};
      \node[datablob] (rdie) [fit=(rtaglabel) (rattributes) (rtag)] {};
    \end{pgfonlayer}

    \path[arrow] (attributes.text east) -- (basedie.west);
    \path[arrow] (attributes.two east) -- (rdie.west);
    \path[valuefrom] (rattributes.text east) -- (countvalue);
    \path[valuefrom] (rattributes.two east) -- (countsize);
  \end{tikzpicture}
  \caption{DIEs describing an array type.}
  \label{fig:array-type-die}
\end{figure}


Array types (\autoref{fig:array-type-die}) are represented by a DIE with
a \verb|DW_TAG_array_type| tag. A \verb|DW_AT_type| attribute contains
a reference to the type of the elements of the array. The space reserved for
attribute values can be insufficient to store the number of elements in the
array, so instead a chain of related DIEs is referenced using
a \verb|DW_AT_sibling| attribute, and one of the elements in the chain is
\verb|DW_TAG_subrange_type| DIE. The latter specifies the \emph{size} —using
a \verb|DW_AT_byte_size| attribute— of the type needed to store the
\emph{count} of items, and the location of the value itself using
a \verb|DW_AT_count| attribute.


\minisec{Type Aliases}

As a convenience, programming languages usually allow defining new user-defined
names for types. This is particularly handy to avoid repeating complex type
declarations in the code of a program. In C type aliases are introduced with
the \mintinline{c}{typedef} keyword:

\begin{minted}{c}
/* "compare_function_t" is an alias to a function pointer type */
typedef int (*compare_function_t) (const void* a, const void*b);
\end{minted}

A type alias (\autoref{fig:typedef-die}) is represented by a DIE with tag
\verb|DW_TAG_typedef|\footnote{The name of the tag hints that
the DWARF format was designed with the C language in mind.} tag. The aliased
type is referenced using a \verb|DW_AT_type| attribute, and
a \verb|DW_AT_name| attribute provides the \emph{name} of the aliased type.

\begin{figure}
  \centering
  \begin{tikzpicture}[die]
    \node[dielabel] (taglabel) {Tag};
    \node[dieitems=1] (tag) {\verb|DW_TAG_typedef|};
    \node[dielabel] (attrlabel) {Attributes};
    \node[dieitems=2] (attributes)
      { \verb|DW_AT_name| \nodepart{two} \verb|DW_AT_type|};

    \node[enumitems=1] (typename) [right=4cm of attrlabel, yshift=-2mm] {\emph{name}};
    \node[datablob] (basedie) [below=0.5cm of typename] {Type DIE};

    \path[valuefrom] (attributes.text east) -- (typename);
    \path[arrow] (attributes.two east) -- (basedie);

    \begin{pgfonlayer}{background}
      \node[datablob] (die) [fit=(taglabel) (attributes) (tag)] {};
    \end{pgfonlayer}
  \end{tikzpicture}
  \caption{DIE describing a type alias.}
  \label{fig:typedef-die}
\end{figure}


\minisec{Structured Types}

\begin{figure}
  \centering
  \begin{tikzpicture}[die]
    \node[dielabel] (taglabel) {Tag};
    \node[dieitems=1] (tag) {\verb|DW_TAG_structure_type|};
    \node[dielabel] (attrlabel) {Attributes};
    \node[dieitems=3] (attributes) {
        \verb|DW_AT_byte_size|
      \nodepart{two}
        \verb|DW_AT_name|
      \nodepart{three}
        \verb|DW_AT_sibling|
    };

    \node[enumitems=1] (sizevalue) [right=4cm of attrlabel, yshift=-6mm] {\emph{size}};
    \node[enumitems=1] (namevalue) [below=5mm of sizevalue] {\emph{name}};

    \path[valuefrom] (attributes.text east) -- (sizevalue);
    \path[valuefrom] (attributes.two east) -- (namevalue);

    \node[dielabel] (rtaglabel) [yshift=-1cm] {Tag};
    \node[dieitems=1] (rtag) {\verb|DW_TAG_member|};
    \node[dielabel] (rattrlabel) {Attributes};
    \node[dieitems=4] (rattributes) {
        \verb|DW_AT_type|
      \nodepart{two}
        \verb|DW_AT_name|
      \nodepart{three}
        \verb|DW_AT_data_member_location|
      \nodepart{four}
        \verb|DW_AT_sibling|
    };

    \node[datablob] (nextmemberdie) [on chain, yshift=-9mm] {Next Member DIE};
    \node (memberellipsis) [on chain, yshift=-9mm] { ... };
    \path[arrow] (rattributes.four south) -- (nextmemberdie.north);
    \path[arrow] (nextmemberdie.south) -- (memberellipsis.north);

    \node[datablob] (basedie) [right=4cm of rattrlabel, yshift=-2mm] {Type DIE};

    \node[enumitems=1] (mname) [below=5mm of basedie] {\emph{membername}};
    \node[enumitems=1] (moffset) [below=5mm of mname] {\emph{offset}};

    \begin{pgfonlayer}{background}
      \node[datablob] (die) [fit=(taglabel) (attributes) (tag)] {};
      \node[datablob] (rdie) [fit=(rtaglabel) (rattributes) (rtag)] {};
    \end{pgfonlayer}

    \path[arrow] (attributes.three south) -- (rdie.north);
    \path[arrow] (rattributes.text east) -- (basedie.west);
    \path[valuefrom] (rattributes.two east) -- (mname);
    \path[valuefrom] (rattributes.three east) -- (moffset);
  \end{tikzpicture}
  \caption{DIEs describing a structured type}
  \label{fig:struct-type-die}
\end{figure}

\noindent Structured types, also known as \emph{record types} or
\emph{compound data types} depending on the programming languages, are
used-defined data structures which contain a collection of related elements,
which can be accessed using a numeric index or (most commonly) by names given
to them. Each element may also be called \emph{field} or \emph{member}. In
C records are defined using the \mintinline{c}{struct} keyword:

\begin{minted}{c}
  struct Point {
    int x;  /* First member  */
    int y;  /* Second member */
  };
\end{minted}

\noindent A record type (\autoref{fig:struct-type-die}) is described using
a set of related DIEs. At the top level, a DIE with tag
\verb|DW_TAG_structure_type| provides the size of the record type in bytes, by
means of a \verb|DW_AT_byte_size| attribute, and the name of the record type
using a \verb|DW_AT_name| attribute.

In order to describe the members, a DIE with tag \verb|DW_TAG_member| is
used for each of them. These DIEs contain a \verb|DW_AT_type| attribute
which references the DIE of the member type, a \verb|DW_AT_name| attribute
with name of the member, an attribute of type
\verb|DW_AT_data_member_location| specifies the \emph{offset} of the member in
memory, counted from the starting address of the record. The member DIEs are
linked from the top level DIE using a chain of \verb|DW_AT_sibling| links.

Some programming languages allow defining \emph{opaque} record types. The
layout and fields of the type are only visible in the program module where the
type is implemented. Such type is defined in C by omitting the specification
of the \mintinline{c}{struct} fields in \verb|.h| a header, and writing its
complete description in the \verb|.c| implementation:

\begin{minted}{c}
  /* File: opaque.h */
  typedef struct _Opaque Opaque;
  ...

  /* File: opaque.c */
  struct _Opaque {
    uint8_t data_blob[100];
    ...
  };
  ...
\end{minted}

In this case, an additional DIE with tag \verb|DW_TAG_structure_type| is
present, \emph{without} specifying the size of the data type with
a \verb|DW_AT_byte_size| attribute. Instead, a \verb|DW_AT_declaration|
attribute is added, flagging the type as declared, but not neccessarily
defined. Optionally, if the complete definition is known by the compiler, it
might add \verb|DW_AT_specification| attribute pointing to the top level DIE
which describes the type in full.


\minisec{Union Types}

Union types use the same structure of DIEs as record types, with a couple of
small differeces: the tag for the top-level DIE is \verb|DW_TAG_union_type|,
and a \verb|DW_AT_byte_size| attribute which indicates the size of the
\emph{largest member type}.


\minisec{Enumeration Types}

The representation of enumerations (\autoref{fig:enum-type-die}) shares
a number of similarities with the representation of record and enumeration
types: a top level DIE, in this case with tag \verb|DW_TAG_enumeration_type|,
describes the size of the integral type needed for values of the enumerated
type using a \verb|DW_AT_byte_size| attribute, and a \verb|DW_AT_name|
attribute contains the name of the type.

The set of values is described by a DIE each, with tag
\verb|DW_TAG_enumerator|. Enumerators contain a \verb|DW_AT_name| attribute
with the name of the enumerator, and its associated value in
a \verb|DW_AT_constvalue| attribute.


\begin{figure}
  \centering
  \begin{tikzpicture}[die]
    \node[dielabel] (taglabel) {Tag};
    \node[dieitems=1] (tag) {\verb|DW_TAG_enumeration_type|};
    \node[dielabel] (attrlabel) {Attributes};
    \node[dieitems=3] (attributes) {
        \verb|DW_AT_byte_size|
      \nodepart{two}
        \verb|DW_AT_name|
      \nodepart{three}
        \verb|DW_AT_sibling|
    };

    \node[enumitems=1] (sizevalue) [right=4cm of attrlabel, yshift=-6mm] {\emph{size}};
    \node[enumitems=1] (namevalue) [below=5mm of sizevalue] {\emph{name}};

    \path[valuefrom] (attributes.text east) -- (sizevalue);
    \path[valuefrom] (attributes.two east) -- (namevalue);

    \node[dielabel] (rtaglabel) [yshift=-1cm] {Tag};
    \node[dieitems=1] (rtag) {\verb|DW_TAG_enumerator|};
    \node[dielabel] (rattrlabel) {Attributes};
    \node[dieitems=2] (rattributes) {
        \verb|DW_AT_name|
      \nodepart{two}
        \verb|DW_AT_const_value|
    };

    \node[datablob] (nextmemberdie) [on chain, yshift=-9mm] {Next Enumerator DIE};
    \node (memberellipsis) [on chain, yshift=-9mm] { ... };
    \path[arrow] (rattributes.four south) -- (nextmemberdie.north);
    \path[arrow] (nextmemberdie.south) -- (memberellipsis.north);

    \node[enumitems=1] (mname) [below=3.7cm of namevalue] {\emph{membername}};
    \node[enumitems=1] (moffset) [below=5mm of mname] {\emph{value}};

    \begin{pgfonlayer}{background}
      \node[datablob] (die) [fit=(taglabel) (attributes) (tag)] {};
      \node[datablob] (rdie) [fit=(rtaglabel) (rattributes) (rtag)] {};
    \end{pgfonlayer}

    \path[arrow] (attributes.three south) -- (rdie.north);
    \path[valuefrom] (rattributes.text east) -- (mname.west);
    \path[valuefrom] (rattributes.two east) -- (moffset.west);
  \end{tikzpicture}
  \caption{DIEs describing an enumeration type}
  \label{fig:enum-type-die}
\end{figure}


\minisec{Functions}

The representation functions using DIEs uses the same structure as record
and enumeration types: the top level entry contains general information about
the function itself, using attributes, plus a link to a chain of DIEs.
Function entries come in different flavors, depending on their tag:

\begin{itemize}
  \item \verb|DW_TAG_entry_point| \hfill \\
    Describes a piece of executable that can be invoked in the same way as
    a function, which does not have entity as a function in the source
    program. For example, this can be used by a \gls{pascal} compiler to
    represent the code inside the \mintinline{pascal}{program} block.
  \item \verb|DW_TAG_inlined_subroutine| \hfill \\
    Describes a function for which the code has
    been copied by the compiler inside another part of the program (i.e.
    \emph{inlined}) as a performance optimization.
  \item \verb|DW_TAG_subroutine| \hfill \\
    Describes normal functions. In \gls{object-oriented} languages is also
    used to describe methods, with the type the method applies to linked as
type DIE of the first parameter. \end{itemize}

The basic structure is shared among the three types, with the following
attributes in the top level DIE:

\begin{itemize}
  \item A \verb|DW_AT_name| attribute provides the name of the function.
  \item A \verb|DW_AT_type| attribute references the return type of the
    function. This is only present if the function returns a value.
  \item A \verb|DW_AT_external| attribute, which contains a flag indicating
    whether the function is available to be used from compilation units other
    than the one where its definition lives. This flag is optional, and only
    required to be present (and have a non-zero value) for functions which
    are referenced from the \verb|.debug_pubnames| ELF section.
\end{itemize}

\begin{figure}
  \centering
  \begin{tikzpicture}[die]
    % Top level DIE
    \node[dielabel] (taglabel)  {Tag};
    \node[dieitems=1] (tag)     {\verb|DW_TAG_subprogram|};
    \node[dielabel] (attrlabel) {Attributes};
    \node[dieitems=4] (attributes) {
      \verb|DW_AT_external| \nodepart{two}
      \verb|DW_AT_name|     \nodepart{three}
      \verb|DW_AT_type|     \nodepart{four}
      \verb|DW_AT_sibling| };
    \begin{pgfonlayer}{background}
      \node[datablob] (die) [fit=(taglabel) (attributes) (tag)] {};
    \end{pgfonlayer}

    % Top level DIE attribute values
    \node[enumitems=1] (ATname)     [right=4cm of attributes, yshift=3mm] {\emph{name}};
    \node[enumitems=1] (ATexternal) [above=5mm of ATname] {\emph{external}};
    \node[datablob]    (ATtype)     [below=5mm of ATname] {Type DIE};

    \path[valuefrom] (attributes.text east)  -- (ATexternal.west);
    \path[valuefrom] (attributes.two east)   -- (ATname.west);
    \path[arrow]     (attributes.three east) -- (ATtype.west);

    % 1st Parameter DIE
    \node[dielabel]   (p1 taglabel) [yshift=-1.5cm] {Tag};
    \node[dieitems=1] (p1 tag)        {\verb|DW_TAG_formal_parameter|};
    \node[dielabel]   (p1 attrlabel)  {Attributes};
    \node[dieitems=4] (p1 attributes) {
      \verb|DW_AT_name| \nodepart{two}
      \verb|DW_AT_type| \nodepart{three}
      ...               \nodepart{four}
      \verb|DW_AT_sibling| };
    \begin{pgfonlayer}{background}
      \node[datablob] (p1 die) [fit=(p1 taglabel) (p1 attributes) (p1 tag)] {};
    \end{pgfonlayer}

    \path[arrow] (attributes.four south) -- (p1 die);

    % 1st parameter DIE attribute values
    \node[datablob]    (p1 ATtype) [right=3.7cm of p1 attributes, yshift=-1mm] {Type DIE};
    \node[enumitems=1] (p1 ATname) [above=5mm of p1 ATtype] {\emph{name}};

    \path[valuefrom] (p1 attributes.text east)  -- (p1 ATname.west);
    \path[arrow]     (p1 attributes.two east)   -- (p1 ATtype.west);

    % Ellipsis to additional parameter DIEs
    \node[datablob] (other die) [on chain, yshift=-1cm] { ... };
    \node[datablob] (pN die) [on chain, yshift=-8mm] {Formal parameter DIE};
    \node (ellipsis) [on chain, yshift=-8mm] { ... };

    \path[arrow] (p1 attributes.four south) -- (other die.north);
    \path[arrow] (other die.south) -- (pN die.north);
    \path[arrow] (pN die.south) -- (ellipsis);

  \end{tikzpicture}
  \caption{DIEs describing a function type}
  \label{fig:function-type-die}
\end{figure}


From the top-level DIE, a \verb|DW_AT_sibling| attribute references the first
entry of the chain, which provides additional information about the function
and its source code. Among all the entries in the chain, the ones with the
\verb|DW_TAG_formal_parameter| describe the parameters of the function. It is
guaranteed that the entries describing the parameters will be in the chain in
the same order as they appear in the source program, but they may not be
consecutive. Each parameter entry contains the following attributes:

\begin{itemize}
  \item A \verb|DW_AT_name| attribute provides the name of the parameter.
  \item A \verb|DW_AT_type| attribute references the DIE for the type of
    the parameter.
\end{itemize}

