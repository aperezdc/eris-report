% vim: ft=tex spell spelllang=en

\chapter{Contextualization}
% TODO: \minitoc ?

Computing would not be understood without the accompanying tools which enable
IT professionals to actually \emph{do something} with computers: from simple
switches and lights in the early times, to the sophisticated programming
languages and tools of the present times, all of them enable humans to
\emph{instruct machines to do things}.

%% TODO

\section{Dynamic Programming Languages}

With  of faster computer


\subsection{Virtual Machines}

\subsection{JIT Compilers}


\subsection{The Lua programming language}

The Lua language is a “powerful, fast, lightweight, embeddable scripting
language”\footnote{\url{http://www.lua.org/about.html}}. It was initially
created as a data description language at
\href{http://www.puc-rio.br/}{PUC-Rio}, and has since evolved into a general
purpose programming language. It has been used in industrial applications
(e.g. Adobe Lightroom) and it has seen widespread usage in the video games
industry.

\section{Binding Native Code to Lua}

\subsection{Lua C API}
	\label{sec:lua-c-api}

% Plain API
% API wrappers

\subsection{Binding Generators}

Binding generators are tools that can be used to create a binding to a library
in an automated way. Often they fall into the category of \glspl{transpiler}:
they take as input the source code of the code to generate a binding for,
and generate a new set of source files which contain the code of the binding.
This set of source files are themselves compiled into a loadable module for
the target programming language or virtual machine, making it a
\emph{build-time} solution.

More than often, binding generators do \emph{not} include a full parser for
the origin programming language, and they require to be fed a simplified
version of the code being wrapped. This can be a nuisance for code bases which
use complex language constructs unsupported by the binding generator.

% \minisec{Example: SWIG}
%
% The most widely known Open Source binding generator is probably
% SWIG\footnote{\url{http://swig.org}}, which supports creating bindings for
% multiple programming languages — since version 1.3.26 Lua is just one of the
% supported targets. SWIG has the ability of reading C/C++ header files as
% input, and it includes a fairly complete C/C++ parser.


\subsection{Foreign Function Interfaces}



% Examples:
%
% GObject-Introspection
% LuaJIT FFI
% Standalone FFI module

\section{Executable Formats}

\subsection{ELF}

The “Executable and Linkable Format” (ELF, formerly called “Extensible Linking
Format”) is a common standard file format for executable programs, object
code, shared libraries, and even core dumps. Since its publication as part of
the System V Release 4 (SVR4) Application Binary Interface (ABI) specification
it has been adopted by many Unix-like (Solaris, most of the BSD variants,
GNU/Linux), and non-Unix operating systems (most notably, OpenVMS, BeOS, and
its successor Haiku).

\subsection{DWARF}

The DWARF \href{http://dwarfstd.org}{specification}


