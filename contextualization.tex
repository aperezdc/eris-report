% vim: ft=tex spell spelllang=en

\chapter{Contextualization}
% TODO: \minitoc ?

Computing would not be understood without the accompanying tools which enable
IT professionals to actually \emph{do something} with computers: from simple
switches and lights in the early times, to the sophisticated programming
languages and tools of the present times, all of them enable humans to
\emph{instruct machines to do things}.

%% TODO


\section{The Lua programming language}

The Lua language is a “powerful, fast, lightweight, embeddable scripting
language”\footnote{\url{http://www.lua.org/about.html}}. It was initially
created as a data description language at
\href{http://www.puc-rio.br/}{PUC-Rio}, and has since evolved into a general
purpose programming language. It has been used in industrial applications
(e.g. Adobe Lightroom) and it has seen widespread usage in the video games
industry.

\section{Executable Formats}

\subsection{ELF}

The “Executable and Linkable Format” (ELF, formerly called “Extensible Linking
Format”) is a common standard file format for executable programs, object
code, shared libraries, and even core dumps. Since its publication as part of
the System V Release 4 (SVR4) Application Binary Interface (ABI) specification
it has been adopted by many Unix-like (Solaris, most of the BSD variants,
GNU/Linux), and non-Unix operating systems (most notably, OpenVMS, BeOS, and
its successor Haiku).


\subsection{DWARF}

The DWARF \href{http://dwarfstd.org}{specification}

